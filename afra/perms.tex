\documentclass{article}

\begin{document}
\section*{Permutations and Combinations (kinda)}
\subsection*{Jinish ta ki}
A permutation is an arrangement of some objects, where sequence matters, a combination is also
an arrangement of some objects, where sequence does not matter.

That means, ABCD and ACBD are different permutations, but they are the same combination, 
mathematically speaking.

\subsection*{Problems, how and why}
Problems from this chapter will consist of asking the candidate to ``find the number of possible
permutations/arrangements if: blah blah blah". To do so we will employ a few strategies.

Keep in mind, \textit{or} means to add and \textit{and} means to multiply, for more clarification
read Lee Peng book.

\textbf{Example}  You are given six letters, A ... G. Find the number of possible permutations 
(arrangements) of these letters if you are to form a word of six letters. 

\textbf{Solution}  For the first letter, from the left we have six choices, for the second letter
we have five and so on. Since these all have to happen at the same time, we will multiple the 
choices we have for each spot.

\begin{center}
	\fbox{6} $\times$ \fbox{5} $\times$ \fbox{4} $\times$ \fbox{3} $\times$ \fbox{2} $\times$ \fbox{1}
\end{center}
where each box is a spot. The number of possible arrangements comes out to be $6 \times 5 \times
4 \times 3 \times 2 \times 1 = 720$.

\subsection*{The P notation}
\textbf{P} is just a way to save us the hassle of writing this times that a bunch of times. The
expansion follows:

\[^n\textrm{\textbf{P}}_r = \frac{n!}{(n-r)!}\]
where $n$ is the number of objects to arrange, and $r$ is the number of spots to arrange them
in. For the rest of this paper, I'll write it as $n$P$r$ as its easier to type.

\textbf{Example}  You are given six letters, A ... G. Find the number of possible arrangements
of these letters if you are to form a word of four letters.

\textbf{Solution}  We have six letters, and four spots to put them, the P notation hence says 6P4.
\begin{center}
	\fbox{6} $\times$ \fbox{5} $\times$ \fbox{4} $\times$ \fbox{3} = 6P4 = 360
\end{center}

\subsection*{Conditions}
With questions that consist of "if this is that" we must modify our simple multiplication rules.

\textbf{Example}  Given the six letters A ... G, find the number of possible arrangements of a four
letter word if the last letter may only be B, C or D.

\textbf{Solution} The last letter can be B or C or D, which means the number of possible last
letters is 1 + 1 + 1 = 3. And since we have filled the last letter, we have one less object to put
in the first spot, and one further less in the second, and so on.

Using the P notation, we have 3 possible last letters, \textit{and} 5 objects remain for the 3 remaining
spots: 5P3 $\times$ 3.
\begin{center}
	\fbox{5} $\times$ \fbox{4} $\times$ \fbox{3} $\times$ \fbox{3} = 5P3 $\times$ 3 = 180
\end{center}

\textbf{Example} Given the six letters A ... G, find the number of possible arrangements of a four
letter word if the first or last letter may be B, C or D, but not at the same time.


\textbf{Solution} Note that there are two possibilities, seperated by \textit{or} which means we
should do the two situations seperately and add the results.

For the first situation, see that there are three possible first letter choices, and then 5 objects
to be arranged into the remaining three spots.
\begin{center}
	\fbox{3} $\times$ \fbox{5} $\times$ \fbox{4} $\times$ \fbox{3} = 5P3 $\times$ 3 = 180
\end{center}
For the second situation, see that there are three possible last letter choices and then 5 objects
to be arranged into the remaining three spots.
\begin{center}
	\fbox{5} $\times$ \fbox{4} $\times$ \fbox{3} $\times$ \fbox{3} = 5P3 $\times$ 3 = 180
\end{center}
Adding the results of the two, we get $180 + 180 = 360$.

\end{document}
