\section{Nuclear physics}
\subsection{The nuclear model of the atom}
\subsubsection{The atom}
The atom consists of a positively charged nucleus.
Negatively charged electrons orbit in ``shells" around this nucleus.

Alpha particles are very massive\footnote{They are only massive in the atomic scale, they still
weigh next to nothing} and positively charged. When these particles were fired at a thin piece of
gold foil at very high speeds, it is seen that most particles pass straight through the gold, 
barely deflected. However, some do bounce back toward the source of radiation and other directions.
This is a result of the positively charged alpha particles being repelled by the like charge of the
nuclei of the gold atoms. Most alpha particles continue on their path as atoms are mostly empty 
space and they travel through that empty space.

\subsubsection{The nucleus}
The nucleus of an atom consists of two types of subatomic particles: the neutron and proton. 
Protons are positively charge and it is because of them nuclei gain their positive charge. Neutrons
have zero charge but have a mass equal to that of a proton. Electrons orbiting the nucleus
have the same charge as a proton, only negative. In atoms with zero charge, the number
of protons equals the number of electrons. The negative and positive charges cancel out.

The loss of electrons from an atom causes the protons to be in surplus, resulting in a positively
charged atom. The gain of electrons in an atom causes the electrons to be in excess, resulting in
a negatively charged atom. Note that these charged states of atoms, called ions, are caused by
absence or presence of electrons, the protons cause not charged atoms, as they do not move.

The nucleon number of an atom, also called the mass number, is the number of particles in the
nucleus of an atom, in other words, the number of neutrons plus the number of protons. The atomic
number or proton number of an atom is simply the number of protons in that atom.

\[A = Z + N\]
where $A$ is the mass number, $Z$ is the proton number and $N$ is the neutron number of an
atom. Seeing this equation we can find the number of protons or neutrons from the mass number of an
atom given we have atleast two of all three values.

For any particle, we can use the following notation to denote its nucleon number and proton number:
\begin{center}
	$^A_Z$X
\end{center}
where X is the symbol of the particle.

An element is a substance with a unique number of protons. It is its number of protons that defines
an element. Atoms may exist of the same element with different mass numbers, yet the same number
of protons in the nucleus. This is a result of there being a different number of neutrons in that
particular atomic nucleus. Such atoms with the same proton number as the element but a different
neutron number is said to be an isotope of that element. The number of neutrons for the same 
element's vary, and so do the number of isotopes per element.

\subsection{Radioactivity}
\subsubsection{Detection of radioactivity}
Radioactivity is caused by release of radiation. This may be in the form of alpha ($\alpha$) or beta
($\beta$) particles or gamma ($\gamma$) radiation. To detect this radiation, we can may use a 
Geiger-Muller tube and counter which detects radioactive particles and records the rate of 
particles detected per unit time, known as the count rate. It is measured in counts per second or
minute. Cloud chambers may also be used to detect alpha radiation, by seeing the streaks formed in
the gas inside the chamber. Spark counters may be used where sparks are formed whenever radioactive
particles pass through electrified metal gauze.

Background radiation is the radiation from the environment to which we are exposed all the time. It
comes from many sources, such as the radioactive gas, radon which is in the air, rocks and 
buildings containing radioactive materials from the earth, our food and drink as what we eat has
taken in nutrients from the ground which contains radioactive materials and finally cosmic rays 
which come from space, mostly from space. 

Before detecting radiation, we must first measure the background radiation in that area. Once we
have done so we will measure the radiation and subtract the background radiation from the reading
gotten as that is the value of the radiation actually being measured.

\subsubsection{The three types of emission}
Emission of radiation is spontaneous, in that it is unpredictable when it will occur or in which
direction it will occur.

Alpha particles consist of a pair of neutrons and a pair of protons. They are positively charged
with a magnitude of two (+2). They are identical to the nuclei of helium atoms.
\begin{center}
	$^4_2\alpha = ^4_2$H
\end{center}
above is the nuclide representation of an alpha particle.
Alpha particles are not particularly penetrating, and they stop movement within only few 
centimetres of atmospheric air. It is very ionising\footnote{The ability to knock electrons out
of orbit, causing absence of electrons, forming ions} because of its large size and relatively slow
speed.

Beta particles are high speed electrons that come from the nucleus, they have a mass of zero and a
charge of -1.
\begin{center}
	$^0_{-1}\beta = ^0_{-1}$e
\end{center}
Beta particles are more penetrating than alpha particles, because of their higher speed. They are
also smaller in size and hence collision with a particle is less likely, as a result they are less
ionising than alpha particles.

Gamma radiation is a high frequency electromagnetic wave. Gamma waves are very fast and hence they
are least likely to interact with a particle, making them the least ionising. But they are the
most penetrating of the three for the same reason.

In an electric field, positively charged alpha particles are attracted to the negative charge,
and beta particles, being negatively charged are attracted to the positive charge. Alpha particles
are less deflected than the beta particles because of its higher mass and hence higher momentum. 
The both are
deflected in opposite directions. The uncharged gamma radiation are unaffected.

The charged alpha and beta particles form a current (rate of movement of charge) when they move. As
a result, a force affects these particles when they move through an magnetic field. This force
causes deflection of the particles, and it followings the left hand rule. Note that, current is
opposite the direction of electron flow. The alpha particles are deflected less, again because of
their higher mass. The gamma radiation is unaffected.

\subsubsection{Radioactive decay}
Radioactive decay is the change in an unstable nucleus resulting in the emission of one of the
three types of radiation. This decay can be shown using decay equations, where the nucleon numbers
for each of the elements sum up to be equal on both sides of the equation, and the same applies
for proton numbers. Given below are two decay equations, showing the release of alpha and beta
radiation from a nuclide: $^{100}_{50}$X into other two nuclides, Y and Z:
\begin{center}
	$^{100}_{50}$X $\rightarrow$ $^{96}_{48}$Y + $^4_2\alpha$

	$^{100}_{50}$X $\rightarrow$ $^{100}_{51}$Z + $^0_{-1}\beta$
\end{center}

\subsubsection{Fission and fusion}
Fusion is the formation of a larger nucleus from the combination of two smaller nuclei, with 
release in energy. Fusion is the process by which energy is released in stars.

Fission occurs when a nucleus recieves a neutron and splits into two or more daughter nuclei, 
alongside release in energy.
