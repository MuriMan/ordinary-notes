\section{Space physics}
\subsection{Earth and the Solar System}
\subsubsection{The Earth}

The Earth is a planet that orbits the Sun once in approximately 365 days. The orbit of the Earth
is an ellipse which is approximately a circle. The Earth also rotates around its own tilted axis,
and it completes one rotation every 24 hours. The Moon is a satellite of the Earth which takes
approximately a month to complete its orbit around Earth. It takes approximately 500 seconds for
light from the sun to reach the Earth.

The average orbital speed for a celestial body can be derived from that of speed:
$$ v = \frac{2\pi r}{T} $$
where $v$ is the average orbital speed, $r$ is the radius of the orbit that has been plugged into
the formula for circumference to find the orbit distance, and $T$ is the time taken for one whole
orbit.

\subsubsection{The Solar System}

The Solar System consists of one star, the Sun around which eight planets orbit. The names of the
planets follow in order of increasing distance from the Sun:
\begin{enumerate}
	\item Mercury.
	\item Venus.
	\item Earth.
	\item Mars.
	\item Jupiter.
	\item Saturn.
	\item Uranus.
	\item Neptune.
\end{enumerate}

There are also minor planets in orbit of the sun, these include dwarf planets such as Pluto. There
is also an asteroid belt in between Mars and Jupiter, consisting of asteroids orbiting the Sun. All
planets have bodies that orbit them, called their moons or satellites. Some bodies such as comets
also exist which orbit the Sun in huge elliptical orbits.

The strength of the gravitational field at the surface of a planet depends on the mass of the 
planet itself. Around the planet, it is directly related to distance from the planet.

The Sun contains most of the mass of the Solar System and the strength of its gravitational field,
at its surface is far greater than those of any planet. For a planet in orbit, it is the attractive
force of this gravitational field that keeps it in orbit. However, with distance from the Sun, the
strength of its gravitational field decreases as the distance from the Sun increases, the same 
hence applies for the planets' orbital speeds as with lower force there is a lower acceleration.

\subsubsection{The Sun as a star}

The Sun is a star of medium size, consisting mostly of hydrogen and helium, which radiates most
of its energy in the infrared, visible and ultraviolet regions of the electromagnetic spectrum.
Stars are powered by nuclear reactions that release energy, in stable ones the nuclear reactions
involve the fusion of hydrogen into helium, shown in the following equation using nuclide notation:

\begin{center}
	\ce{^{1}_{1}H + ^{1}_{1}H -> ^{2}_{2}He}
\end{center}

\subsubsection{Stars}

Galaxies are made up of many billions of stars. The Sun is a star in the galaxy known as the Milky
Way. The other stars that consist the Milky Way are far farther from the Earth than the Sun is.
Kilometres are not enough to measure these distances, the light-year is the unit of astronomical
distance, which is the distance travelled by a ray of light in one year. That is:

$$ (\num{3.0e8})(365)(24)(60)(60) \approx {9.4608 \times 10^{15}}\SI{ }{m} $$
is one light-year.

Stars form from intersteller clouds of gas and dust that contain hydrogen called molecular clouds.
A protostor is an interstellar cloud collapsing and increasing in temperature as a result of
its internal gravitational attraction. A protostar becomes a stable star when the inward force of
gravitational attraction is balanced by an outward force due to the high temperature in the centre
of a star. All stars eventually run out of hydrogen, the fuel for the internal nuclear fusion 
reaction. Most stars expand to form red giants and more massive ones expand to form red supergiants
when most of the hydrogen in the centre of a star has been converted to helium. A red giant from a
less massive star forms a planetary nebula with a white dwarf at its centre. A red supergiant
explodes as a supernova, forming a nebula containing hydrogen and new heavier elements, leaving
behind a neutron star or a black hole at its centre. The nebula from a supernova may form new
stars with planets that orbit them.

\subsubsection{The Universe}

The Milky Way is one of the many billions of galaxies making up the Universe and the diameter of
the Milky Way is approximately $ 100 000 = \num{e6} $ light-years. A redshift is the increase in
the observed wavelength of the electromagnetic radiation emitted from receding stars and galaxies.
The light from distant galaxies shows redshift and that the further away the galaxy, the greater
the observed redshift and the faster the galaxy's speed is away from the Earth.

How does redshift provide evidence for the Big Bang Theory??????? HELP
