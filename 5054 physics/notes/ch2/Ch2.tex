\section{Thermal physics}
\subsection{Kinetic particle model of matter}
\subsubsection{States of matter}

Matter is composed of particles and can be found in three states: solid, liquid and gas. Their 
characteristics follow:
\begin{itemize}
	\item Solids:
		Solids are rigid, and have a fixed shape and volume. They cannot be squashed.
	
	\item Liquids: 
		Liquids have no rigidity, nor shape, in fact they take the shape of the container they are
		placed into. They are incompressible in that they cannot be squashed and hence have fixed
		volumes.
	
	\item Gases: 
		Gases are not rigid, nor do they have a fixed shape, but they take the shape of their 
		container. They are compressible and hence can be squashed. They may expand to take the
		shape of the container they were put into.
\end{itemize}

The temperature of matter dictates its state. Cooling reduces the kinetic energies of the particles
getting them closer to solids and vice versa. The following diagram shows the changes of state of
matter:

\begin{center}
\ce{
	Solid <=>[Melting][Freezing] Liquid <=>[Boiling][Condensing] Gas
}
\end{center}

\subsubsection{Particle model}
All matter is made of particles, examples of which are ions, atoms molecules, electrons etc. The
matter in their three states have these properties related to their particles:

\begin{itemize}
	\item Solid: Have a regular arrangement of particles, which all have relatively low kinetic
		energy. Their movement consists only of vibration about fixed points. The hotter the
		solid, the faster that vibration.

	\item Liquid: 
		The particles are not arranged in any pattern, and they are farther from each
		other than in solids. These particles can move by sliding around each other, while also
		vibrating. The hotter the liquid, the faster the vibrations of these particles.
	\item Gas: The particles in gases have are widely separated from one another, they are not
		in contact whatsoever but may collide amongst each other. The particles move freely about
		very energetically, bouncing off of each other and their container.

\end{itemize}

The higher the temperature of a particle, the higher their kinetic energy. The two are directly
related. There is a temperature, $-273 \SI{}{\celsius}$, where the particles in matter are not
moving at all and have zero kinetic energy.

In a gas, the particles are ever-moving and ever-colliding. For a gas in a container, there are
three variables, its temperature $T$, its volume $V$ and its pressure $p$. Note the following cases:


\begin{center}
Given constant $V$, $p \propto T$.

Given constant $p$, $V \propto T$.

Given constant $T$, $p \propto \frac{1}{V}$\footnote{None of these three statements are concrete
mathematically, these have been written to provide a sense of the relationships of the variables
in a qualitative sense.}.
\end{center}

With variance in pressure and volume as a result of a varied temperature, the product of the first
two variables will remain equal before and after the change in temperature. In other words:

$$ p_1 V_1 = p_2 V_2 $$

\subsection{Thermal properties and temperature}
\subsubsection{Thermal expansion of solids, liquids and gases}

Under heat, all three states of matter will undergo expansion, i.e., their volume will increase.
This property is used in liquid in glass thermometers where the liquid expands according to the
temperature it has been subjected to.

This expansion is a result of the particles in each state of matter gaining kinetic energy and
moving about more intensely, causing themselves to push each other apart. Solids expand the
least, liquids expand more than solids and gases expands the most per unit temperature increase.

The temperature $\SI{-273}{\celsius}$ has been mentioned before. This temperature is called
absolute zero. It is from this temperature another scale for temperature is started, the Kelvin
scale. To convert between the two scales:
$$ T \textrm{ (in K)} = \theta \textrm{ (in \SI{}{\celsius})} + 273 $$

\subsubsection{Specific heat capacity}

An increase in the temperature of an object increases its internal energy and vice versa. This is
because the average kinetic energies of all the particles in the object has increased.

The specific heat capacity of a material is the energy required per unit mass per unit temperature
increase, it is a numerical value unique to each material and is denoted by $c$:

$$ c = \frac{\Delta E}{m \Delta \theta} $$
where $\Delta E$ is the change in internal energy of the material, $m$ is its mass and $\Delta \theta$
is the change in its temperature.

\subsubsection{Melting, boiling and evaporation}

Melting, freezing, boiling and condensation are all energy transfers without changes in temperature.

Every substance boils and melts at fixed temperature, for water the temperatures are \SI{0}{\celsius}
and \SI{100}{\celsius} respectively.

Evaporation is the escape of more energetic particles from a surface of a liquid into gas. This
occurs on the surface of the liquid below its boiling point. Evaporation rate is affected directly
by temperature, surface area exposed to heat and movement of air over the body of liquid. In other
words, the more the magnitude of these three variables, the faster the evaporation.

Latent heat is the energy required to change the state of a substance. During changes in state,
the temperature of the substance is constant for some time before changing once again. At this
constant temperature, not all particles of the substance have changed states and the substance is
a mixture of its two states. During this time, energy is being absorbed by the substance to change
its state. It is this energy that is the latent heat for that substance.

\subsection{Transfers of thermal energy}
\subsubsection{Conduction}

Heat is conducted through a non-metal as increased vibrations of particles at one end collides
with the other end. As these collisions keep on happening, the kinetic energy of the particles in
the substance increase steadily. In this case, non-metals are thermal insulators.

Metals have free (delocalised) electrons which can move about. When they are heated, not only do
their particles vibrate more intensely, the electrons move about all over the metal. These 
electrons pass along heat to the extents of the metal. This is why metals conduct better than 
non-metals.

\subsubsection{Convection}

When one side of a fluid\footnote{Liquid or gas.} is heated, the particles in that side gain
kinetic energy and get further apart. As a result, the density of that region lowers. The hotter
and hence less dense particles rises through the fluid, pushing the denser and cooler particles
downward. These particles also become less dense while the particles previously pushed above are
pushed back down. This cycle is called the convection current. This is what happens in pans and
pots and stuff.

\subsubsection{Radiation}

Thermal energy may be transferred by infrared radiation, which is a form of electromagnetic wave.
This form of radiation needs no medium to propagate. Infrared radiation is invisible to the naked
eye and can be detected by our nerves.

Shiny white surfaces tend to reflect off all infrared radiation they are subjected to and hence
they are the worst absorbers.

Matte black surfaces will absorb all infrared radiation they are subjected to and hence are the
worst reflectors. As they are the best absorbers they are also the best emmitters.

\subsubsection{Consequences of thermal energy transfer}

Insulators are used to maintain the internal temperature of buildings. Heaters are placed at the
bottom of a room and cooling machines are placed at the top to set up convection currents
respective to the densities of the particles undergoing temperature change.
