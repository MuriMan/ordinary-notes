\section{Atoms, elements and compounds}
\subsection{Elements, compounds and mixtures}
An element is a substance, every atom of which has the same proton number. A mixture is two or
more substances which are not chemically bonded together. A compound is two or more elements
chemically bonded to each other.

\subsection{Atomic structure and the Periodic Table}
The structure of an atom consists of a central nucleus and protons surrounded by electrons in 
shells. A proton has a relative charge of +1 and an electron has that of $-1$. A neutron has
no charge.

The proton number/atomic number of an atom is the number of protons in the nucleus of an atom, every element
has a unique proton number.

The mass number/nucleon number of an atom is the total number of protons and neutrons in the 
nucleus of an atom.

For elements with proton numbers inclusively between 1 and 20, the first electron shell holds 2
electrons and all the rest hold 8. The electronic configuration of an atom is the number of 
electrons in each shell outward. For example an atom with 8 electrons will have the following
configuration: 2, 6.

Atoms in Group VIII of the periodic table have a full outer shell. The number of outer shell 
electrons is equal to the group number in Groups I to VII. The number of occupied electron shells
equals the period number of the element.

\subsection{Isotopes}
Isotopes are atom sof the same element having same number of protons but different numbers of
neutrons. Isotopes of the same element have same chemical properties as they have same
electronic configuration. 

Atoms are represented as $^Y_X$Z, where $Y$ is the mass number, $X$ is the proton number and Z
is the atom's symbol.

The relative atomic mass of an atom can be found using the abundance of its isotopes:
$$
\textrm{average mass} = \frac{m_1 p_1 + m_2 p_2 + ...}{100}
$$
where $m_n$ is the mass of the isotope with abundance $p_n$. 

\subsection{Ion and ionic bonds}
Positive ions are caused by lack of electrons in an atom, called cations. Negative ions are caused
by an excess of electrons in an atom, called anions.

The lattice structure of ionic compounds consists of a regularly arranged alternative cations
and anions. An ionic bond is a strong electrostatic attraction between oppositely charged ions.

Electrostatically bonded high melting and boiling points because of this bond which takes a lot of 
heat energy
to break. They are conductive in aqueous solutions as the ions become separated and can move around
in aqueous solutions, i.e., they have free charged particles in aqueous solution.

\subsection{Simple molecules and covalent bonds}
