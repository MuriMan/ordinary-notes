\section{Chemistry of the environment}
\subsection{Water}

\textbf{Tests for water.} Water can be tested for using hydrated and anhydrous forms of two salts,
copper(II) sulfate, \ce{CuSO4} and cobalt(II) chloride, \ce{CoCl2}. Water turns anhydrous
copper(II) sulfate blue, and anhydrous cobalt(II) chloride pink.

\textbf{Purity of water.} The purity of any substance can be determined from its actual boiling
point or melting point, and the melting point it shows. Water melts at \SI{0}{\celsius} and
boils at \SI{100}{\celsius}.

In practical chemistry, distilled water is used in place of tap water as tap water contains many
impurities. Water from natural sources may contain impurities such as:
\begin{itemize}
	\item Dissolved oxygen.
	\item Metal compounds.
	\item Plastics.
	\item Sewage.
	\item Harmful microbes.
	\item Nitrates from fertilisers.
	\item Phosphates from fertilisers and detergents.
\end{itemize}
Some of these substances, may be beneficial. Dissolved oxygen is useful for aquatic life, as they
use it to aerobically respire and some of the metal compounds are essential minerals for life.

Yet some of these are harmful too. Certain metal compounds are toxic, and the plastics are harmful
for aquatic life. Sewage contains harmful microbes that cause disease and nitrates and phosphates
lead to deoxygenation of water and damage to aquatic life.\\

\textbf{Treatment of the domestic water supply.} Tap water sent to homes is treated by authorities
before being distributed. The water from natural sources is sedimented and filtered to remove
unwanted solids. The tastes and odours of this water is removed using carbon. Finally, the
water is chlorinated to kill any potentially harmful microbes.

\subsection{Fertilisers}

Ammonium salts and nitrates are used as fertilisers. NPK fertilisers are those which contain and
provide the elements nitrogen, potassium and phosphorus for improved plant growth.

\subsubsection{Air quality and climate}

\textbf{Composition of air.} The composition of clean, dry air consists of 78\% gaseous nitrogen
(\ce{N2}), 21\% oxygen (\ce{O2}) and the remaining 1\% is a mixture of noble gases and carbon 
dioxide.

\textbf{Pollutants and sources.} Air is polluted by the substances carbon dioxide, carbon monoxide, methane,
oxides of nitrogen and sulfur dioxide. Carbon dioxide and carbon monoxide are produced when 
carbon containing fuels are completely and incompletely combusted, respectively. Methane forms as
a result of the decomposition of vegetable matter and waste gases of digestion in animals. Oxides
of nitrogen form in car engines and sulfur dioxide forms when fossil fuels contain sulfur are 
burnt.

\textbf{Pollutants and effects.} High levels of carbon dioxide and methane in the atmosphere leads 
to global warming, which leads to climate change. Carbon monoxide is a gas that is toxic to humans.
Certain particulates in the air increase the risk of respiratory problems and cancer. Oxides of
nitrogen cause acid rain, photochemical smog and respiratory problems. Sulfur dioxide causes
acid rain.

Carbon dioxide and methane are greenhouse gases, as they increase the adversity of the greenhouse
effect. This occurs as these gases absorb thermal energy from the sun and inhibit the reflection
and emission of this thermal energy. This reduces the thermal energy lost back into space after
it is recieved from the sun, increasing the average temperatures around the globe.

\textbf{Reducing adverse effects.} Climate change can be hampered by planting trees, reducing
livestock farmed, decreasing use of fossil fuels, increasing use of hydrogen and renewable energy
such as solar panels. Acid rain can be prevented by using catalytic converters in vehicles, 
reducing sulfur dioxide emissions by using low sulfur fuels and flue gas desulfurisation using
calcium oxide.

In a catalytic converter, carbon monoxide and nitrogen monoxide produced inside the car engine
are made to form carbon dioxide and gaseous nitrogen, removing them. The reaction follows
\begin{center}
	\ce{2CO + 2NO -> 2CO2 + N2}
\end{center}

Photosynthesis is a reaction in plants, where, using energy from light, carbon dioxide and water
are reacted to form glucose and oxygen.
\begin{center}
	\ce{6CO2 + 6H2O -> C6H12O6 + 6O2}

	\ce{carbon dioxide + water -> glucose + oxygen}
\end{center}
