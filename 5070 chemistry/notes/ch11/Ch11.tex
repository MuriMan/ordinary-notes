\newcommand\setpolymerdelim[2]{\def\delimleft{#1}\def\delimright{#2}}
\def\makebraces[#1,#2]#3#4#5{%
	\edef\delimhalfdim{\the\dimexpr(#1+#2)/2}%
	\edef\delimvshift{\the\dimexpr(#1-#2)/2}%
	\chemmove{%
		\node[at=(#4),yshift=(\delimvshift)]
		{$\left\delimleft\vrule height\delimhalfdim depth\delimhalfdim
	width0pt\right.$};%
	\node[at=(#5),yshift=(\delimvshift)]
	{$\left.\vrule height\delimhalfdim depth\delimhalfdim
width0pt\right\delimright_{\rlap{$\scriptstyle#3$}}$};}}
\setpolymerdelim()

\section{Organic chemistry}
\subsection{Formulae, functional groups and terminology}
Structural formulae are unambigious descriptions of the way atoms in a molecule are arranged,
showing double bonds between carbon atoms and each molecule bonded to each carbon, eg., 
\chemfig{CH_2=CH_2}, \ce{CH3CH2OH}.

Displayed formulae show all the atoms and bonds between them in a molecule. An example, in the form
of methane follows:
\begin{center}
\chemfig{C(-[:0]H)(-[:90]H)(-[:180]H)(-[:270]H)}
\end{center}

The general formulae of homologous series follow:

\begin{itemize}
	\item Alkanes: \ce{C_nH_{2n+2}}.
	\item Alkenes: \ce{C_nH_{2n}}.
	\item Alcohols: \ce{C_nH_{2n+1}OH}.
	\item Carboxylic acids: \ce{C_nH_{2n+1}COOH}.
\end{itemize}

Structural isomers are compounds with same molecular formulae but different structural formulae.
For example, \ce{C4H8} has two structural isomers: \chemfig{CH_3CH=CHCH_3} and 
\chemfig{CH_2=CHCH_2CH_3}.

A functional group is an atom or group of atoms that determine the chemical properties of a 
homologous series.

All members of a homologous series has the following properties:
\begin{itemize}
	\item Have the same functional group.
	\item Have the same general formula.
	\item Differ by the next member by one \ce{-CH_2-} unit.
	\item Display a trend in physical properties.
	\item Sharing similar chemical properties.
\end{itemize}

A saturated organic compound is that in which all carbons are singly bonded, in other words, all
carbon-carbon bonds in a saturated compound are single bonds. An unsaturated organic compound is
that which has atleast one carbon-carbon double bond.

\subsection{Naming organic compounds}

The number of carbon atoms in a molecule of an organic compound determines its name, so does the
homologous series it is a part of. The pattern is:
\begin{center}
	\begin{tabular}{c c}
		number of carbons & name \\\hline
		1 & meth- \\
		2 & eth- \\
		3 & prop- \\
		4 & but- \\
	\end{tabular}
\end{center}
The next in the series follow the geometric pattern of nomenclature (five-penta; six-hexa; ...).

The end of the name of an organic compound follows, which depends on the homologous series it is a
part of:
\begin{center}
	\begin{tabular}{c c}
		homologous series & name \\\hline
		alkane & -ane \\
		alkane & -ene \\
		alcohol & -ol \\
		carboxylic acid & -oic acid \\
	\end{tabular}
\end{center}

So a carboxylic acid with four carbons is called \textit{butanoic acid}.

Alcohols and carboxylic acids react to form compounds called esters (see Section 11.7). They are
composed of one alkyl part and another acidic part, and their nomenclature comes down to being
\textit{alkyl acidoate}. An ester formed from ethanol and butanoic acid will be called ethyl 
butanoate. The general structural and displayed formula for esters follow:

% \chemfig{C(-[:0]H)(-[:90]H)(-[:180]H)(-[:270]H)}
\begin{center}
	\chemfig{R-C(=[:45]O)(-[:315]O-R')}
\end{center}

\begin{center}
	\chemfig{RCOOR'}
\end{center}
where R is carbon side of the acid part and \chemfig{R'} is the carbon side of the alcohol part.

\subsection{Fuels}

There are three fossil fuels: coal, natural gas and petroleum. Methane is the main constituent of
natural gas.

Hydrocarbons are organic compounds containing carbon and hydrogen only. Petroleum is a liquid mixture
of
such hydrocarbons. All of these hydrocarbons all have different uses and they must be separated.
Fractional distillation is used to do so, using a fractionating column that separates the 
constituent hydrocarbons by rising them up depending on their properties. 

So, as we observe the
products from the bottom to the top of the fractionating column:
\begin{itemize}
	\item Compound chain length decreases.
	\item Compound volatility increases.
	\item Boiling points lower.
	\item Viscosity\footnote{Viscosity is the resistance of flow of a fluid.} lowers.
\end{itemize}

The topmost fraction that is separated from petroleum is refinery gas, which is used for heating
and cooking. Petrol, also called gasoline, is the next, used as motor fuel. Subsequently, naphtha
is collected which is used as a chemical feedstock. Paraffin is used as a fuel in jet engines and
heeting oil. Diesel oil is used as fuel in diesel engines which trucks and large vehicles use. Fuel
oil is used in ships and home heating. Lubricating oil is used to make waxes and polishes. Lastly,
the bottommost fraction is called bitumen, which is used to surface roads.

\subsection{Alkanes}

The bonding between all atoms in alkanes is single covalent, and hence they are saturated compounds
\footnote{Note that they are saturated due to the carbons being singly bonded, the bonds between
other atoms are not of importance in this case.}. Alkanes are generally unreactive compounds, 
except when combusted and substituted by chlorine.

In a substitution reaction, a single atom or group of atoms is replaced by another atom or group
of atoms.

The substitution of alkanes by chlorine is a photochemical reaction, where the activation energy
is provided by ultraviolet light. An example of the reaction of chlorine and methane follows:

\begin{center}
	\schemestart
		\chemfig{
			C(-[:0]H)(-[:90]H)(-[:180]H)(-[:270]H) 
		}
		+
		\chemfig{Cl-Cl}
		\ce{->}
		\chemfig{
			C(-[:0]H)(-[:90]H)(-[:180]H)(-[:270]Cl) 
		}
		+
		\chemfig{H-Cl}
	\schemestop
\end{center}

\begin{center}
	\ce{CH4 + Cl2 -> CH3Cl + HCl}
\end{center}
where the products are chloromethane, and hydrogen chloride.

\subsection{Alkenes}

The bonding in alkenes includes a carbon-carbon double bond, and hence they are unsaturated
hydrocarbons. Alkenes are produced by the \textit{cracking} of longer chained alkanes, with 
hydrogen as a byproduct, in presence of a catalyst at high a high temperature of around 
\SI{500}{\celsius}. For example:
\begin{center}
	\ce{C10H22 ->[heat][catalyst] 5C2H4 + H2}
\end{center}

The fractional distillation of petroleum produces alkanes, but not enough to meet consumer demands.
This is why catalytic cracking is performed on large chain alkanes, to meet the demand of the
shorter chained alkenes.

Aqueous bromine reacts with unsaturated compunds, losing its brown colour, this reaction does
not happen with saturated compounds, however. The reaction of bromine with an unsaturated compound,
ethene, follows:
\begin{center}
	\schemestart
		\chemfig{
			% C(-[:0]H)(-[:90]H)(-[:180]H)(-[:270]H) 
			C(-[:90]H)(-[:270]H)=C(-[:90]H)(-[:270]H)
		}
		+
		\chemfig{Br-Br}
		\ce{->}
		\chemfig{
			C(-[:90]H)(-[:270]H)(-[:180]Br)-C(-[:90]H)(-[:270]H)(-[:0]Br)
		}
	\schemestop
\end{center}
\begin{center}
	\schemestart
		\chemfig{
			CH=CH
		}
		+
		\chemfig{Br_2}
		\ce{->}
		\chemfig{
			CHBrCHBr
		}
	\schemestop
\end{center}
where the product is called dibromoethane.

The above is an example of an addition reaction, that which is where two or more reactants form one
product. Alkenes undergo two more addition reactions.

\textbf{Hydrogen in presence of nickel.} An alkene will react with hydrogen in heated conditions,
(150-300\SI{}{\celsius}), in presence of a nickel catalyst, to form a corresponding alkane.
\begin{center}
	\schemestart
		\chemfig{
			% C(-[:0]H)(-[:90]H)(-[:180]H)(-[:270]H) 
			C(-[:90]H)(-[:270]H)=C(-[:90]H)(-[:270]H)
		}
		+
		\chemfig{H-H}
		\ce{->[Ni][150-300\SI{}{\celsius}]}
		\chemfig{
			C(-[:90]H)(-[:270]H)(-[:180]H)-C(-[:90]H)(-[:270]H)(-[:0]H)
		}
	\schemestop
\end{center}
\begin{center}
	\schemestart
		\chemfig{
			CH=CH
		}
		+
		\chemfig{H_2}
		\ce{->[Ni][150-300\SI{}{\celsius}]}
		\chemfig{
			CH_3CH_3
		}
	\schemestop
\end{center}

\textbf{Steam in presence of phosphoric acid.} An alkene reacts with steam at a temperature of
\SI{300}{\celsius}, at a pressure of \SI{6000}{kPa} 
in presence of phosphoric acid as a catalyst to form the corresponding alcohol.
\begin{center}
	\schemestart
		\chemfig{
			% C(-[:0]H)(-[:90]H)(-[:180]H)(-[:270]H) 
			C(-[:90]H)(-[:270]H)=C(-[:90]H)(-[:270]H)
		}
		+
		\chemfig{H-OH}
		\ce{->[\ce{H3PO4}][\SI{300}{\celsius}, \SI{6000}{kPa}]}
		\chemfig{
			C(-[:90]H)(-[:270]H)(-[:180]H)-C(-[:90]H)(-[:270]H)(-[:0]OH)
		}
	\schemestop
\end{center}
\begin{center}
	\schemestart
		\chemfig{
			CH=CH
		}
		+
		\chemfig{H_2O}
		\ce{->[\ce{H3PO4}][\SI{300}{\celsius}, \SI{6000}{kPa}]}
		\chemfig{
			CH_3CH_2OH
		}
	\schemestop
\end{center}

\subsection{Alcohols}

Ethanol is an alcohol that can be produced by two methods.

\textbf{Fermentation of glucose.} Aqueous glucose can be fermented at 25-30 \SI{}{\celsius} in the
presence of yeast and in the absence of oxygen. The reaction is:
\begin{center}
	\ce{C6H12O6 ->[yeast][enzymes] 2C2H5OH + 2CO2}
\end{center}

\textbf{Catalytic addition of steam.} see the end of Section 11.5.

The ethanol produced by fermentation comes from a renewable source, in this way, when the ethanol
is used as a fuel it can be said to be ``carbon neutral". It uses relatively simple large 
apparatus. The ethanol in this method is made by a batch process, where the process needs to be
started anew each time. It is a relatively slow process, and the ethanol produced must be
subsequently purified through distillation.

The ethanol by cat-hydration comes from a non-renewable source in the form of petroleum. It requires
small scale apparatus, which can withstand pressure. The ethanol is produces in a continuous
reaction with a high reaction rate, yielding very pure ethanol. However, it is a sophisticated and
complex method.

Alcohols completely combust, i.e., react with oxygen\footnote{Any reaction with oxygen is said 
called burning in oxygen.}, to form water and carbon dioxide. When they combust incompletely, they
form carbon monoxide and water.

\begin{center}
	\ce{C2H5OH + 7/2O2 -> 2CO2 + 3H2O}
\end{center}
is an example of complete combustion.
\begin{center}
	\ce{C2H5OH + 5/2O2 -> 2CO + 3H2O}
\end{center}
is an example of incomplete combustion.

Ethanol is used as a solvent for paints, glues, perfumes, aftershaves and printing inks. The
combustion of ethanol produces a lot of heat, which can be used in engines as fuel when ethanol
is mixed with petrol.

\subsection{Carboxylic acids}

All carboxylic acids have a \chemfig{-COOH} group, which is what gives each acid its acidic 
property as it is this group that disassociates to give the hydrogen ion:
\begin{center}
	\ce{COOH -> COO- + H+}
\end{center}

\textbf{Reactions with metals.} As with acids, the reaction of carboxylic acids with metals forms
a metal salt and hydrogen gas.

\begin{center}
	\ce{2RCOOH + 2$\Lambda$ -> 2RCOO$\Lambda$ + H2}
\end{center}
where $\Lambda$ is a metal, R is a placeholder for the remaining part of the carboxylic acid.

\textbf{Reactions with bases.} As with acids, the reactions of carboxylic acids with bases forms
a metal salt and water.

\begin{center}
	\ce{RCOOH + $\Lambda$OH -> RCOO$\Lambda$ + H2O}
\end{center}

\textbf{Reactions with carbonates.} As with acids, the reactions of carboxylic acids with bases 
forms a metal salt, water and carbon dioxide.
\begin{center}
	\ce{RCOOH + $\Lambda$CO3 -> RCOO$\Lambda$ + H2O + CO2}
\end{center}
Ethanoic acid can be formed from ethanol by two methods. 
When ethanol reacts with an oxidising
agent, it forms ethanoic acid. Bacteria may also oxidise vinegar, which is ethanol to form ethanoic
acid:

\begin{center}
	\ce{C2H5OH + 2[O] -> CH3COOH + H2O}
\end{center}

When carboxylic acids and alcohols react in presence of sulfuric acid as a catalyst, an ester and
water is formed.

\begin{center}
	\schemestart
	\chemfig{R-C(=[:45]O)(-[:315]O(-H))}
	+
	\chemfig{R-O-H}
	\ce{<=>[H2SO4]}
	\chemfig{R-C(=[:45]O)(-[:315]O-R')}
	+
	\chemfig{H-OH}
	\schemestop
\end{center}
\begin{center}
	\schemestart
	\chemfig{RCOOH}
	+
	\chemfig{R'OH}
	\ce{<=>[H2SO4]}
	\chemfig{RCOOR'}
	+
	\ce{H2O}
	\schemestop
\end{center}

\subsection{Polymers}

A polymer is a large molecule built up from smaller constituent molecules, which are called 
monomers. The process by which monomers form polymers is called polymerisation, which is of two
types.

\textbf{Addition polymerisation.} In addition polymerisation, the monomers join up to form the
polymer and nothing else, which is a form of addition reaction and hence the name. Alkenes, under
high pressure, heat and a catalyst will react with each other to form a polymer. An example with
ethene is given, where $n$, a very large number of ethene molecules react to give an $n$-long chain
of ethene called poly(ethene).

\begin{center}
	\ce{\textit{n} (\chemfig{CH_2=CH_2})}
	\ce{->[high pressure][heat, catalyst]}
	\chemfig{\vphantom{CH_2}-[@{op,.75}]CH_2-CH_2-[@{cl,0.25}]}
	\polymerdelim[height = 4pt, indice = \!\!n]{op}{cl}
\end{center}
The \chemfig{CH_2=CH_2} is called the repeat unit of poly(ethene), because its repetition is what
forms poly(ethene). Note the loss of \chemfig{C=C} after polymerisation, which applies to any and
all alkenes undergoing the process.

\textbf{Condensation polymerisation.} In condensation polymerisation, water is formed usually, as
the byproduct. There are two condensation polymers to study: polyamides and polyesters. The amine
functional group is \chemfig{NH_2-}; that for carboxylic acids is \chemfig{-COOH} and alcohols have
\chemfig{-OH} as their functional group. For polymers formed through condensation, there are three
different monomers:

\begin{center}
	\chemfig{N(-[:135]H)(-[:225]H)-R-N(-[:45]H)(-[:315]H)}
\end{center}
is a diamine (two amine groups).

\begin{center}
	\chemfig{C(=[:135]O)(-[:225]O(-[:180]H))-R-C(-[:45]O)(-[:315]O-H)}
\end{center}
is a dicarboxylic acid (two carboxyl groups).

\begin{center}
	\chemfig{H-O-R-O-H}
\end{center}
is a diol (two alcohol groups).

% \begin{center}
% 	\chemfig{C(=[:135]O)(-[:225]O(-[:180]H))-C(-[:90]H)(-[:270]R)-N(-[:45]H)(-[:315]H)}
% \end{center}
% is an amino acid.

In the above, the R is a variable carbon containing group. Note that, when reacting, a carboxylic
acid group loses \chemfig{-OH}, and the alcohol and amine groups lose \chemfig{-H}.

\textbf{Polyamides.} The polymerisation of dicarboxylic acids and diamines form polyamides, with
amide linkages.
