\section{Chemical reactions}
\subsection{Physical and chemical changes}
In a physical change, the substances present remain chemically the same and no new substances
are formed. Such changes are often easy to reverse.

Changes such as melting and boiling are endothermic as heat is taken in, changes such as 
condensing and freezing are exothermic as heat is given out.

Chemical changes are those where a new substances is formed. Most chemical changes happen through
chemical reactions, almost all of which are exothermic, very few are endothermic.

\subsection{Rate of reaction}
A reaction progresses when effective collisions occur between reactants to form a molecule of a
product. The rate of effective collisions is hence the rate of the reaction. This rate of effective
collisions can be influenced. These effective collisions can only occur if the colliding particles
have a minimum energy ($E_a$).

Increasing the number of particles per unit volume means there are more particles per unit volume,
when that is the case the particles are more likely to react.

Increasing the kinetic energy (KE) of particles by applying heat makes them move faster, meaning
more particles are likely to collide more frequently and with energy greater that $E_a$.

A catalyst is that which increases the rate of a reaction, decreases the $E_a$ and is unchanged
at the end of the reaction.

Changing solution concentration changes particles per unit volume, refer above.

Changing gas pressure does the same, refer above.

Increasing surface area of solids means there are more exposed surfaces with which reactants can
effectively collide, meaning effective collisions are more likely. The opposite is also true.

Increasing temperature increases KE of particles, refer above.

Addition of a catalyst, including enzymes, increases reaction rate and decreases $E_a$.

Investigating the rate of a reactioon comes down to measuring the rate at which reactants are used
up or the rate at which products are formed. The information can be represented in a graph:

\begin{center}
\begin{tikzpicture}
    % Axes
    \draw[->] (0,0) -- (6,0) node[right] {Time};
    \draw[->] (0,0) -- (0,4) node[above] {Loss in Mass};

    % Curve A (slower and levels off later)
    \draw[thick] (0,0) .. controls (2,2.5) and (4,3.2) .. (5.5,3.5) node[below, midway] {A};

    % Curve B (faster and levels off earlier)
    \draw[thick,dashed] (0,0) .. controls (1,3) and (3,3.4) .. (5.5,3.5) node[above, midway] {B};
    
\end{tikzpicture}
\end{center}
In reaction B, the reactants were powdered, with increased surface area and hence the reaction had
a faster rate than A with just solid reactants. Notice that they loss in mass is equal but the rate
of loss of mass is the difference.

\subsection{Reversible reactions and equilibrium}
Most reactions are one-way, i.e., reactants react to form products and that's it. Some reactions,
are reversible, in such reactions, reactants react to form products and those products also react
to form reactants. Such reactions are shown with a ``$\rightleftharpoons$":

\begin{center}
	\ce{N$_2$(g) + 3H$_2$(g) <=>[forward reaction][backward reaction] 2NH$_3$(g)}
\end{center}

Changing the direction of a reversible reaction can depend on the conditions applied to it.
Such is the case for hydrating and making anhydrous salts.

\begin{center}
	\ce{CoCl2 * 6H2O <=>[heat][water] CoCl2 + 6H2O}

	\ce{CuSO4 * 6H2O <=>[heat][water] CuSO4 + 6H2O}
\end{center}

Note that, \ce{CuSO4} is white while the hydrated form \ce{CuSO4 * H2O} is blue. Hydrated cobalt (II)
chloride is pink whereas the anhydrous form is blue.

A closed system is that where no reactants or products can escape from the reacting system. In such
a system, a reversible reaction is in equilibrium when the rate of forward reaction equals the
rate of backward reaction and the concentrations of reactants and products are not changing. The
reagents are reacting but the concentrations do not change.

The position of equilibrium tells us the rate of forward reaction compared to backward reactions.
That means, if the position of equilibrium is to the right, the rate of forward reaction is greater
than that of the backward reaction and vice versa.

Note that, when a change is made to the conditions of a system in dynamic equilibrium, the system 
moves so as to oppose that change.

Changing temperature affects the reaction depending on the enthalpy of the reaction. Changing the
temperature will cause the equilibrium to shift in the direction that will reverse that change.
Increasing the temperature in a reaction where the forward reaction is exothermic will shift 
equilibrium to left, as the system will oppose the change by increasing rate of endothermic 
reaction to lower temperature. The opposite is also true.

Changing pressure only affects reactions whose reagents are all gaseous. Increasing pressure moves
equilibrium toward the side which has less gaseous moles. The opposite is also true.

Increasing concentration of reactants moves equilibrium to right and more products are formed and
vice versa.

Using a catalyst does not affect equilibrium position, but the speed at which the system reaches 
equilibrium is affected by catalyst.

Below is the symbol equation for the Haber process.
\begin{center}
	\ce{N2(g) + 3H2(g) <=>[450 $^{\circ}$C][200 atm] 2NH3(g)} $\Delta H < 0$
\end{center}
The nitrogen is gotten from fractional distillation of liquid air and hydrogen is gotten from 
cracking of crude oil. The above reaction requires an iron catalyst.

In the Contact process, sulfur dioxide is converted into sulfur trioxide.
\begin{center}
	\ce{2SO2(g) + O2(g) <=>[450 $^{\circ}$C][2 atm] 2SO3(g)} $\Delta H < 0$
\end{center}
The Contact process requires a vanadium (V) oxide, \ce{V2O5} catalyst. Sulfur dioxide is gotten
from burning or roasting sulfide ores:
\begin{center}
	\ce{S(s) + O2(g) -> SO2(g)} 
\end{center}
and oxygen is gotten from air.

The conditions in the above industrial processes are such that to optimise economical costs and
safety.

\subsection{Redox}
Roman numbers are used to indicate the oxidation number of an element in a compound (iron (II), 
vanadium (V), etc.).

A redox reaction is where reduction and oxidation occurs simultaneously.

The oxidation of a compound is its gain of oxygen, loss of electrons or increase in oxidation 
number.

The reduction of a compound is its loss of oxygen, gain of electrons or decrease in oxidation
number.

Rules for identifying oxidation number:
\begin{itemize}
	\item It is zero in uncombined states (B, Mg).
	\item It is the same as the charge on an ion (B$^{+2}$).
	\item Sum of oxidation numbers in a compound is zero.
	\item Sum of oxidation numbers in an ion is equal to the charge in the ion.
\end{itemize}
Acidified potassium manganate(VII), \ce{KMnO4}, is an oxidising agent with a purple colour, which it
loses when it is added to a reducing agent. Aqueous potassium iodide, \ce{KI}, is a reducing agent 
which turns brown in presence of an oxidising agent.

An oxidising agent is that which oxidises another substance and is itself reduced. A reducing agent
as a substance that reduces another substance and is itself oxidsed.

Using changes in oxidation numbers, oxidising and reducing agents can be identified.

