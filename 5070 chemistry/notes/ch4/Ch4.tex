\section{Electrochemistry}
\subsection{Electrolysis}

Electrolysis is the decomposition of an ionic compound, in molten or aqueous form, by passage
of electric current.

A simple electrolytic cell consists of two electrodes, the cathode and anode, connected to either
side of a battery and an electrolyte. The anode is positively charged whereas the cathode is
negatively charged. The electrolyte is the aqueous or molten substance that undergoes decomposition
through electrolysis.

In the external circuit consisting of wires connected to the batteries, electrons flow from the
battery to the cathode, making the cathode negatively charged. Here electrons are lost to 
substances in the electrolyte which are oxidised. The anode, being positively charged, attracts
the negative ions from the ionic compound, oxidising them, by taking their
excess electrons electrons. Electrons are
gained at the anode from negative ions, and are lost at the cathode to positive ions. The 
electrodes can be made of any conductive material, such as metals or graphite. Graphite electrodes
are said to be inert as they do not take part in the reaction. Note that, oxidation occurs at
anodes and reduction occurs at cathodes.

In case of the substance, lead (II) bromide, \ce{PbBr2}, the following happens 
during electrolysis with inert electrodes
at either electrode:

\begin{center}
at anode: \ce{2Br- -> Br2 + 2e-}

at cathode: \ce{Pb$^{2+}$ + 2e- -> Pb}
\end{center}

It is observed that a silvery solid accumulates at the cathode, i.e. lead (II) and a red-brown
gas is given off at anode. Note that, from here we can judge that for molten ionic compounds, the
metal part of the compound will always form at the cathode, and the non metal will form at the
anode.

During electrolysis of concentrated aqueous sodium chloride, using inert electrodes, the following
reactions occur at the electrodes:
\begin{center}
	at anode: \ce{2Cl- -> Cl2 + 2e-}

	at cathode: \ce{H+ + 2e- -> H2}
\end{center}
It is observed that a colourless gas is given off at cathode and a green gas is given off at the
anode.

The reason why the metal, i.e., sodium is not discharged at the cathode is because it is not the
least reactive amongst the cations present in the electrolyte. The ions present in the electrolyte
are: \ce{H+}, \ce{Na+}, \ce{Cl-} and \ce{OH-}. Of these, at the cathode the ion discharged is
the least reactive of the cations, and at the anode the ion discharged is the ion that loses 
electrons more readily, and is hence easier to discharge. The reactivity series of cations and
the prefential discharge rule of anions are given below:

$$\ce{Na+} > \ce{Mg^{2+}} > \ce{Al^{3+}} > \ce{Zn^{2+}} > \ce{H+} > \ce{Cu^{2+}} > \ce{Ag+} \footnote{where the ion to the left of $>$ is more reactive than the ion to the right.
}$$
$$ \ce{SO4^{2-}} < \ce{NO3^{-}} < \ce{OH-} < \ce{Cl-} < \ce{Br-} < \ce{I-} \footnote{where the
ion to the left of $<$ is less likely to be discharged than that to the right.}$$

During the electrolysis of dilute sulfuric acid, the ions in the electrolyte are: \ce{H+}, \ce{OH-},
\ce{SO4$^{2-}$}. The only cation, \ce{H+} is released at the cathode and, by preferential rule,
\ce{OH-} is the one which is oxidised at anode.
\begin{center}
	at cathode: \ce{2H+ + 2e- -> H2}

	at anode: \ce{4OH- -> O2 + 2H2O + 4e-}
\end{center}
Colourless gases are given off at both electrodes, but the volume of oxygen produced is half that
of hydrogen. This is a result of the fact that, the mole ratio of oxygen to hydrogen in the 
electrolyte is 1:2.

Consider the case of the electrolysis of aqueous copper (II) sulfate using inert electrodes. 
Uncharged copper (\ce{Cu}) has a pinkish-orange colour, whereas copper ions, \ce{Cu$^{2+}$} have
a blue colour. The following happens at the electrodes in this electrolysis:
\begin{center}
	at cathode: \ce{Cu$^{2+}$ + 2e- -> Cu}

	at anode: \ce{OH- -> O2 + 2H2O + 4e-}
\end{center}
Notice that, the copper ions are reduced to copper, reducing the number of copper ions in 
electrolyte. It is observed that the blue colour of solution fades and a pinkish orange solid
accumulates on the cathode.

An active electrode is that which takes part in the electrolytic process. Consider the case
of the electrolysis of aqueous copper (II) sulfate with copper electrodes. It is seen that copper
deposites on the cathode and copper is dissolved at the anode.
\begin{center}
	at cathode: \ce{Cu$^{2+}$ + 2e- -> Cu}

	at anode: \ce{Cu -> Cu$^{2+}$ +  2e-}
\end{center}

This property of active electrodes and metals comes in use during electroplating, which is a
process of electrolysis in which a metal object is coated/plated with a layer of another metal.

To electroplate a metal object, we must place it as the cathode, and have the anode be the metal with
which we want to plate the metal object. The electrolyte must be the solution of a salt with the
metal to plate with.

\subsection{Hydrogen-oxygen fuel cells}

It is possible to use chemical reactions to produce electrical energy. Everyday batteries work in
this way. A more efficient method would be using a fuel cell. A hydrogen oxygen fuel cell uses
the following chemical reaction

\begin{center}
	\ce{2H2(g) + O2(g) -> 2H2O(l)}
\end{center}
Here, the hydrogen is considered to be a non-polluting fuel. 

As a fuel, hydrogen is more efficient than any other, and the only waste product formed is water.
However, due to it being difficult to produce and unsafe to store, it has failed to gain precedence.
