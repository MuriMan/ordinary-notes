\section{Experimental techniques and chemical analysis}
\subsection{Experimental design}
TODO.

\subsection{Acid-base titrations}
An acid base titration is an experimental method to find the concentration of an acid or alkali. It
involves the use of a pH indicator, a burette and a volumetric pipette. The procedure follows:
\begin{enumerate}
	\item Wash burette with distilled water and the chemical to be used.
	\item Wash pipette with distillied water and the chemical to be used.
	\item Wash conical flask with distilled water.
	\item Fill burette with chemical of known concentration, note reading on burette.
	\item Take known volume chemical of known concentration using volumetric pipette into a conical
		flask. Put two drops of indicator into the conical flask.
	\item Titrate (open tap of burette) burette's liquid into conical flask into conical flask
		until indicator changes colour.
	\item Note the reading on the burette, and subtract initial reading from this. The result is
		the titre volume.
	\item Repeat until concordant values\footnote{Readings with a difference of $\pm \SI{0.1}{cm^3}$.}
	\item Average volumes gotten, which gives titre volume.
\end{enumerate}

The titre volume is that needed to just neutralise the used volume of the substance with unknown
concentration. Using molar ratio, we can find the moles needed for one substance to react with the
other. Since we know volume of both and concentration of one, we can derive concentration of 
other by mole calculation.

\subsection{Chromatography}

Paper chromatography is used to separate mixtures of soluble substances using a suitable solvent.
These substances must be coloured, and they travel down a filter paper at varying distances
depending on how soluble they are in a solvent or how attracted they are to the filter paper. The
ratio of the distance travelled on the chromatogram by the solvent to that travelled by a certain
soluble substance is called its $R_f$ value. As such:

$$ R_f = \frac{\textrm{distance moved by the substance}}{\textrm{distance moved by the solvent front}} $$

For a given solvent at a given temperature, the $R_f$ value is the same for a particular substance,
which can be used to identify the substance.

Colourless substances can be separated by use of \textit{locating agents}. These are substances
that react with colourless substances to give a coloured spot or spots that glow under ultraviolet
light.

\subsection{Separation and purification}

For a mixture of substances, one of which is soluble in a particular soluble and the other isn't,
we can separate the substances by dissolving it into this solvent. 

We may then filter out the 
insoluble substance by making use of filter paper, where the insoluble substance will collect as
residue, and the soluble part will pass through to form the filtrate.

For a given salt solution, we can form crystals of the salt by the following procedure:
\begin{enumerate}
	\item Heat solution in evaporating dish until it becomes saturated, which can be seen using a
		glass rod dipped into the solution. If crystals form on the rod, the solution is saturated.
	\item Cooling this saturated solution will form hydrated crystals of the salt.
	\item These crystals can be dried using paper towels or tissues.
\end{enumerate}

For a substance with a dissolved solid, the liquid into which the solid was dissolved usually has
a much lower boiling point. To separate the two, we take the solution in a round bottom flask, with
a thermometer attached at one side, and a condenser at the other. The thermometer measures the
temperature of the flask, which, when equal to the boiling point of the substance, will cause it
to boil into the condenser. The condenser has cold water running around a tube, which causes the
substance to condense and this condensed liquid is collected in a vessel at the end of the 
condenser.

A solution of different liquids which all have different boiling points can be separated using
fractional distillation. The substance is taken in a round bottom flask which heats it. The 
substance with the lowest boiling point first boils into the fractionating column which is a glass
columnn filled with glass beads which maximise surface area to condense the liquid with the higher
boiling point which may have evaporated. The liquid whose boiling point the temperature is at
the passes through the condenser and collects in a vessel.

A pure substance has a fixed boiling point. The further the boiling point of a substance from its
fixed one, the less pure it is.

\subsection{Identification of ions and gases}
The anion in a substance can be tested with the following procedures:
\begin{itemize}
	\item Carbonate, \ce{CO3^{2-}}: Addition of hydrochloric acid effervesces colourless gas, which
		is carbon dioxide\footnote{Which can be tested with limewater.}.
	\item Chloride, \ce{Cl-}; Bromide \ce{Br-}; Iodide \ce{I-} (in solution): Acidification with 
		dilute nitric
		acid and subsequent addition of silver nitrate gives white, cream and yellow precipitates
		respectively.
	\item Nitrate, \ce{NO3^{-}}: Addition of aqueous sodium hydroxide and aluminium foil, with
		subsequent warming will produce ammonia gas\footnote{To be testeed with moist red litmus.}.
	\item Sulfate, \ce{SO4^{2-}}: Adding barium nitrate to acidified sulfate solution gives white
		precipitate of barium sulfate.
	\item Sulfite, \ce{SO3^{2-}}: Adding aqueous potassium manganate (VII) solution to acidified
		sulfite solution decolourises the purple potassium permanganate solution.
\end{itemize} 

Reactions with sodium hydroxide give the following observations on the following cations:
\begin{itemize}
	\item Aluminium, \ce{Al^{3+}}: Gives white precipitate, which dissolves in excess to give a
		colourless solution.
	\item Ammonium, \ce{NH4+}: Gives off ammonia gas on warming, produces no precipitate.
	\item Calcium, \ce{Ca^{2+}}: White precipitate which is insoluble in excess reagent.
	\item Chromium (III), \ce{Cr^{3+}}: Green precipitate which dissolves in excess sodium 
		hydroxide.
	\item Copper (II), \ce{Cu^{2+}}: Light blue precipitate forms, insoluble in excess.
	\item Iron (II), \ce{Fe^{2+}}: Green precipitate, insoluble in excess. Turns brown near the 
		surface (rusts) when left to stand.
	\item Iron (III), \ce{Fe^{3+}}: Red brown precipitate forms, insoluble in excess.
	\item Zinc, \ce{Zn^{2+}}: White precipitate that dissolves in excess NaOH to give a colourless
		solution.
\end{itemize}

Reactions with aqueous ammonia give the following observations on the following cations:
\begin{itemize}
	\item Aluminium, \ce{Al^{3+}}: Gives white precipitate, insoluble in excess.
	\item Ammonium, \ce{NH4+}: No reaction.
	\item Calcium, \ce{Ca^{2+}}: No reaction (very slight white precipitate).
	\item Chromium (III), \ce{Cr^{3+}}: Green precipitate, insoluble in excess.
	\item Copper (II), \ce{Cu^{2+}}: Light blue precipitate forms, soluble in excess to give a
		dark blue solution.
	\item Iron (II), \ce{Fe^{2+}}: Green precipitate, insoluble in excess. Turns brown near the 
		surface (rusts) when left to stand.
	\item Iron (III), \ce{Fe^{3+}}: Red brown precipitate forms, insoluble in excess.
	\item Zinc, \ce{Zn^{2+}}: White precipitate that dissolves in excess \ce{NH3 (aq)} to give a 
		colourless solution.
\end{itemize}

The following gases can be tested as follows:
\begin{itemize}
	\item Ammonia, \ce{NH3}: Turns damp red litmus paper blue.
	\item Carbon dioxide, \ce{CO2}: Makes limewater (\ce{CaCO3 (aq)}) cloudy.
	\item Chlorine, \ce{Cl2}: Bleaches damp litmus paper.
	\item Hydrogen, \ce{H2}: Makes a pop sound with a lighted split.
	\item Oxygen, \ce{O2}: Relights a glowing splint.
	\item Sulfur dioxide, \ce{SO2}: Decolorouses acidifed aqueous potassium permangante.
\end{itemize}

Non transition metals do not form coloured compounds. Their identities can be analysed using a
flame test since they all burn to give differently coloured flames.

A nichrome wire is heated up in a Bunsen burner before being dipped into concentrated acid. This
wire is then dipped into the salt to be tested and the colour of the flame is obserbed. Salts with
their respective flame colours follow:

\begin{center}
\begin{tabular}{  || c | c || }
	\hline
	Lithium, \ce{Li+} & Red \\\hline
	Sodium, \ce{Na+} & Yellow \\\hline
	Potassium, \ce{K+} & Lilac \\\hline
	Calcium, \ce{Ca^{2+}} & Orange red \\\hline
	Barium, \ce{Ba^{2+}} & Light green \\\hline
	Copper (II), \ce{Cu^{2+}} & Blue green \\\hline
\end{tabular}
\end{center}
