\section{Stoichiometry}
\subsection{Formulae}

All elements have symbols. These elements bond to form molecules, which can also be denoted by
symbols. 

The molecular formula of a compound shows the numbers of and the types of different
elements in a molecule. \ce{C6H12O6} is an example of such a formula, that of glucose. 
The empirical formula of a compound is the simplest whole number ratio of the different atoms or
ions in a compound. \ce{CH2O} is the empirical formula of glucose.

The overall charge in a molecule tends to be zero, unless stated otherwise. The numbers of atoms
of each element and their corresponding charge can hence be deduced. Consider the case of a 
molecule of aluminium and chlorine, \ce{Al^{3+}} and \ce{Cl-} respectively. Each atom of aluminium
has a charge of $+3$, to cancel out which three chlorine atoms, each of charge $-1$ is required.
See that, $3 + 3(-1) = 0$.

Chemical reactions can be denoted by means of symbol equations. An example follows:
\begin{center}
	\ce{C6H12O6 (s) + 6O2 (g) -> 6CO2 (g) + 6H2O (l) }
\end{center}
The above shows two reactants, \ce{C6H12O6} and \ce{O2} and two products, \ce{CO2} and \ce{H2O}.

Note that, the number of atoms of each element on either side of the ``\ce{->}" symbol is equal.
This is said to be a balanced symbol equation. The letters inside of the parentheses represent the
state of matter in which each reactant and product is in, specifically: (s), solid; (l) liquid;
(g) gas.

\subsection{Relative masses of atoms and molecules}

A carbon-12 atom has 6 protons and 6 neutrons. It is this isotope of carbon that is held as the
standard for the atomic masses. That is, each proton and neutron, since both have masses with
negligible difference, is said to have a mass of 1/12th of a carbon-12 atom. The relative atomic
mass of an element is the average atomic mass of all the isotopes of the element, with respect to
the amount of the isotope present. The relative atomic mass of an element is denoted $A_r$.

Consider the case of chlorine, 25\% of all the world's chlorine has an atomic mass of 37, and the
remaining 75\% has an atomic mass of 35. Thus:
$$ A_r = (75\%)(35) + (25\%)(37) = 35.5$$

The relative molecular mass, $M_r$ of a substance is the some of all the relative atomic masses of its
component elements.

\subsection{The mole and the Avogadro constant}
The mole (mol, in short), is the unit of amount of substance. Each mole contains $N_A = \num{6.02e23}$
particles which are atoms, ions or molecules. $N_A$ is the Avogadro constant or Avogadro's number.

Understand that, the $A_r$ of a substance is its molar mass, which is the grams of mass per mole
of that substance. Knowing this,

$$ \textrm{amount of substance (mol)} = \frac{\textrm{mass (g)}}{\textrm{molar mass (g/mol)}} $$
Notice that the units cancel out to give (mol).

For gases, each mole of a gas has a volume of \SI{24}{dm^3} at room temperature and pressure.

The concentration of a substance is the amount of it per unit volume. It can be measured as grams
of substance per unit volume or moles of substance per unit volume. To convert between the two
units, divide by molar mass and multiply molar mass to go from \SI{}{g/dm^3} and \SI{}{mol/dm^3}
and back, respectively:
$$ c \textrm{ (moldm$^{-3}$)} = (c \textrm{ (gdm$^{-3}$)})(M_r)$$
hence,
$$c \textrm{ (gdm$^{-3}$)}  =  \frac{c \textrm{ (moldm$^{-3}$)}}{M_r}$$

The litre is the unit for measuring liquids. It is a measure of volume and is equivalent to the
cubic decimetre (\SI{}{dm^3}). Small amounts of liquids are measured in \SI{}{cm^3}, equivalent to
the millilitre (ml). To convert from cubic decimetre and cubic centimetre, multiple by $1000 = 10^3$
and vice versa.

Given the masses of element in a compound, or the percentage of each element per molecule of 
compound, we can derive the empirical formula of the compound. If we know the $M_r$ of the compound,
the molecular formula of the compound can be found.

Consider the case of a molecule where 33\% of it is carbon and the rest is oxygen.
