\section{Acids, bases and salts}
\subsection{The characteristic properties of acids and bases}

Aqueous solutions of acids contain \ce{H+} ions and those of alkalis contain \ce{OH-} ions. An
acid is hence a proton donor and base is a proton acceptor. Bases are metal oxides or hydroxides,
and alkalis are soluble bases, amongst these, alkalis are those which are soluble.\\

Acids react with metals, bases and carbonates.

\textbf{Reaction with metals.} Let $\Lambda$ be a metal that is more reactive than hydrogen (see
Chapter 9) and H$\Delta$ an acid with the anion $\Delta^-$.
\begin{center}
	\ce{$\Lambda$ + H$\Delta$ -> $\Lambda\Delta$ + H2}
\end{center}

\textbf{Reaction with bases.} Acids undergo neutralisation reactions with bases, where the
corresponding salt and water is formed.

\begin{center}
	\ce{$\Lambda$OH + H$\Delta$ -> $\Lambda\Delta$ + H2O}
\end{center}
