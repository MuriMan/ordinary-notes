\section{Classification}
\subsection{Concept and use of a classification system}

A living thing is defined by its ability perform the following:
\begin{itemize}
	\item Movement is the ability of an organism to change position or place.
	\item Respiration is the chemical process through which organisms gain energy (see Section 10).
	\item Sensitivity is an organisms ability to detect changes in their environment and respond
		to them.
	\item Growth is defined as the permanent increase in size and dry mass of an organism
	\item Reproduction means making more of the same organsism
	\item Nutrition is done by taking in materials to build cells, provide energy for metabolism,
		etc.
	\item Excretion is the removal of waste and excess products from the body.
\end{itemize}

All organisms can be classified into specific groups by their common features. A species is a group
of organism that can reproduce to produce fertile offspring.

The binomial naming system is a system of naming species as an internationally agreed system, in
which the scientific name of an organism is made up of two parts, namely the genus and species
in the following format:
\begin{center}
	\textit{Genus species}
\end{center}
Note that, the \textit{Genus} has a capitalised first letter and \textit{species} is all in lower 
case.

Dichotomous keys consist of features of organisms corresponding to the organisms features.

\subsection{Features of organisms}

The largest groups living organisms are classified into are called kingdoms. There are five: animal,
plant, fungus, protoctist and prokaryote.

Animals have cells with nuclei, lacking cell walls and chloroplasts. They feed on organic 
substances\footnote{Organic substances, in terms of chemistry, are any substances that contain
the element carbon, but in biology the term will be used to refer to compounds made by living 
things} made by other living organisms.

Plants have cell walls made of cellulose and often contain chloroplasts. They make their own
food by use of photosynthesis, a majority of them have roots, stems and leaves, yet some lack
these structures.

Fungi are organisms made up of microscopic thread like structures called hyphae. Hyphae are composed
of several cells joined end-to-end. Such cells have cell walls but those are not made of cellulose.
Such organisms are decomposers, in that they break down waste material from other organisms and
dead organisms. They reproduce by means of spreading spores, which are cells with a tough, 
protective outer covering, which can be spread by an animal, or the wind. Fungi are multicellular,
yet some may be unicellular. They lack chlorophyll, and feed by digesting waste material.

Some protoctists have plant like cells while others have animal like cells. Most protoctists are
unicellular but some are multicellular.

Prokaryotes are organisms such as bacteria. They are usually unicellular, lacking proper nuclei,
have cell walls made not of cellulose, lacking mitochondria, and circularly organised DNA, freely
suspended in the cytoplasm. They often have additional rings of DNA called plasmids.

The animal kingdom is further classified into two groups: vertebrates and invertebrates.

Vertebrates are animals with backbones, this group includes fish, amphibians, reptiles, birds
and mammals.

Mammals are animals with hairy skin, their young develop in their uterus, where they are attached
to their mother through a placenta. Mothers have mammary glands which produce milk on which the
young feed. They have different kinds of teeth: incisors, canines, premolars and molars. They
have an ear flap on the outside of their bodies. They have sweat glands and a diaphragm, a muscle
which assists in breathing (see Section 9).

Birds have feathers, and sometimes scales. They have beaks and their front two limbs are wings,
though not all birds can use them to fly. They lay eggs with hard shells to reproduce.

Reptiles are vertebrates with scaly skin which lay eggs with soft shells.

Amphibians are vertebrates which lack scales, their eggs are laid in the water and lack shells.
The tadpoles (babies) livie in the water but the adults often live on land. Tadpoles have gills
for gas exchange underwater, but adults have lungs.


Fish are vertebrates with scaly skin, with gills that they have throughout their life cycles,
with fins to swim and lay eggs lacking shells in the water.

Arthropods are invertebrates with an exoskeleton and several pairs of jointed legs.

Myriapods are made up of several similar segments, each of which has jointed legs and a pair of
antennae.

Insects are arthropods with three pairs of jointed legs, and two pairs of wings, of which a pair
may be vestigial in that they have evolved to be so small that they are now useless. They breathe
through tubes called trachae. Their body can be divided into a head, thorax and abdomen. They have
a pair of antennae.

Arachnids have four pairs of jointed legs, lack antennae and have their body consisting of two 
parts, namely the cephalothorax and abdomen.

Crustaceans are arthropods with more than four pairs of jointed legs with two pairs of antennae.

Ferns are plants with roots, stems and leaves (fronds), which reproduce by means of spores 
produced on the undersides on their lives. They do not flower.

Flowering plants are those with roots, stems and leaves, whose reproduction consists of flowers and
seeds where those seeds are produced in ovaries inside the flower. This group can be further 
divided into two, dicots and monocots. 

Dicots produce seeds with two cotyledons, with a main root with side stemming out of it. Their leaves
have a network of veins, and the flowers are arranged in multiples of four or five. They have
vascular bundles in their stems, arranged in a ring (see Section 7).

Monocots produce seeds with a single cotyledon, where their roots grow out directly from the stem.
The leaves have parallel veins and the flower parts in multiples of three. The vascular bundles
in monocots are arranged randomly.

Viruses are not considered to be living organisms as they do not show any living properties until
they enter a living cell. They consist of ``cells" which are made up of a protein coat and genetic
material, composed of ribonucleic acid, RNA.
