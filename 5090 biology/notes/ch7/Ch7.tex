\section{Transport in flowering plants}
\subsection{Uptake and transport of water and ions}

Plants take in water from the soil, through root hairs. Root hairs are long and thin, and hence
have a large surface area, through which uptake of water and mineral ions can be more and faster.
These root hairs are on the edge of the root, a little farther, separated by the root cortex lies
a xylem vessel.

The xylem vessels are long hollow tubes made from dead cells joined end to end. The walls of xylem
are made of cellulose and lignin, the latter of which is very strong, helping keep plants upright.
Through these xylem vessels, water arrives in leaves through visible veins on the leaf under which
the xylem vessels are. The water reaches specifically the mesophyll of the leaves.

The movement of water through these tubes is seen by placing the root of a plant into coloured
water. That coloured water will be absorbed by the root cells and will be brought up through the
stem and into the leaves of the plant. When the plant is cut, it will be seen that there are coloured
tubes going across it.

\subsection{Transpiration and translocation}

Transpiration is the loss of water vapour from a plant.

As discussed before, the xylem vessels reach the plant in the spongy mesophyll layer as part of
the vascular bundle, from where
water diffuses out by osmosis into the cells. As a result, these spongy mesophyll cells tend to
have a thin film of moisture over them. Some of this moisture evaporates from the surface of the
cells into the air spaces in the spongy mesophyll layer. From these air spaces the water diffuses
out into the atmosphere through stomata. 

To replace this lost water, water travels
from the xylem into the mesophyll cells. The lack of water in the spongy mesophylls of leaves 
causes a net lower pressure at the top of xylem vessels, where the leaves are. This causes water
to flow up the xylem vessels. This movement of water up the leaves due to differences in pressure
is called the transpiration stream. The difference in pressure itself is known as the transpiration
pull. The transpiration pull pulls up a column of water molecules through the xylem, the water
molecules themselves are held together by intermolecular forces of attraction.

When the wind speed in the atmosphere is high, transpiration can occur more as the water in the
atmosphere is quickly moved away. This increases the water potential gradient between the inside
and outside of the leaf. Water, as a result, diffuses out faster from inside the leaves. This
increases rate of transpiration.

Humidity, i.e., the water content in the air, affects rate of transpiration as it does the water
potential between the inside and outside of the leaf. The less the humidity, the faster the 
rate of transpiration and vice versa.

When water is lost from the cells faster than water is absorbed into the plant, wilting occurs
as the turgidity of the cells pushing against each other is lost.

Translocation is the transport of sucrose and amino acids in the phloem from parts of plants that
produce or release them, called sources, to parts of plants that use or store them, called sinks.
Phloem are tubes, similar to xylem only that they are not hollow nor are they made from hollowed
cells. They are also part of vascular bundles. Phloem can transport the substance they are 
transporting up or down the plant whereas xylem can only transport the substances upwards towards
the leaves.
