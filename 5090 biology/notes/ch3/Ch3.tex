\section{Movement into and out of cells}
\subsection{Diffusion and osmosois}

Water is used as a solvent in many organsisms.

All substances are made up of particles called atoms or molecules. These particles are in a state
of constant movement, the intensity of this movement depends on their kinetic energy which can
be controlled using the temperature of the particles in question. This movement results in particles
spreading out passively, requiring no extra energy, and this energy of particles is the energy
by which substances diffuse.

The movement of a substance occurs from one region to another.

Diffusion is the net movement of molecules or ions from a region of their higher concentration
that of their lower concentration, i.e., down a concentraion gradient as a result of their random
movement. The surface area in contact between the substances being diffused increases diffuse rate
with increase in itself, and vice versa. The temperature of the particles corresponds to the 
particles' kinetic energy, where the more the energy the faster the diffusion and vice versa.
The concentration gradient is the difference between the concentration of the substance diffusing
between the regions of diffusion. The higher this gradient, the faster the diffusion and vice
versa. Lastly, the closer the two regions are, the faster the diffusion and vice versa.

Osmosis is the net movement of water molecules from a region of higher water potential to a region
of lower water potential, through a partially permeable membrane.

Plants are supported by the pressure of the water inside the vacuole pressing outwards on the cell
wall.

Osmosis is the process by which water moves in and out of animal cells. When an animal cell has
a lower water potential than its surroundings, water moves into the cell, eventually as the water
moves in the strain becomes large enough to make the cell burst. When the initial conditions are
opposite, a significant amount of water moves out of the cell. This causes the cell to shrink.

In case of plant cells, when an excess amount of water has entered the cell, the cell becomes 
tight and firm, as the cell will not burst due to the cell wall. It is said that the cell has become
turgid. The pressure with which the cell is kept turgid is called turgor pressure. However, when
excess water moves out of the cell, the cell shrinks, but the cell wall cannot, as a result the
cell membrane tears away from the cell wall. This condition of the cell is known as plasmolysis.
This damages the cell wall and may result in the cell dying.

\subsection{Active transport}
Active transport is the movement of molecules or ions into or out of a cell through a cell
membrane, from a region of their lower concentration to that of their higher concentration, using
energy released during respiration.

This occurs in root hair cells of plants, where mineral ions are higher in concentration outside
the root hair cell than inside. As a result, energy from respiration is used to conduct these cells
into the plant's cell.
