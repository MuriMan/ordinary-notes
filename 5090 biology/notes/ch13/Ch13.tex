\section{Excretion}
\subsection{Excretion}
Excretion is the removal of toxic materials and the waste products of metabolism from organisms. In
mammals, carbon dioxide and urea are waste products of respiration and deamination, excreted 
through lungs and urine respectively.

\subsection{Urinary system}
The urinary system consists of the kidneys, ureters, bladders and urethra.

The kidneys remove urea, excess salts and water from the blood and form urine from these toxic
substances. The urine then flows to the ureters into the bladder. It is in the bladder the urine
is stored until the sphincter muscle to the urethra is opened and the urine is excreted through it.

Note that, urea is only exreted due to the toxicity of it.

The cross section of the kidney is made up of two parts, pyramid shaped medulla and the cortex. The
ureter joins the kidney at the medulla. The kidneys are made up of tubules called nephrons.

Nephrons begin in the cortex, loop into the medulla and back out and again into the medulla to join
up with the ureter.

Blood flows into the kidney through the renal artery, which devides to form many capillaries
called glomeruli. As the blood flows through the glomerulus in the Bowman's capsule, which is the
first part of the nephron, small molecules are filtered out of it, including water, glucose, urea
and other ions. Not all of these filtered substances need to be lost, so the glomerulus continues 
along the nephron as the nephron drops down into the cortex as the loop of Henle. Glucose, water
and some ions are reabsorbed into the blood from the loop of Henle into the blood vessels that 
surround it, known as reabsorption. The liquid remaining in the nephrons is urine, consisting of
urea, excess water and excess ions and can hence be excreted.

Urea is produced in the liver. Amino acids go to the liver after they have been digested through
the hepatic portal vein, here they are assimilated by the liver. Proteins cannot be stored and 
those that are not needed immediately are broken down. This breakdown consists of removal of the
nitrogen-containing part of the amino acid, forming urea. The process through which urea is formed,
i.e., the breakdown of amino acid in the described way is called deamination.

