\section{Disease and immunity}
\subsection{Disease}
A pathogen is any microorganism that causes disease. A disease that can pass from one host to 
another is called a transmissible disease. This transmission can occur in two modes: directly or
indirectly.

\textbf{Direct transmission} occurs when an infected person comes in direct contact with an 
uninfected person. This may be in the form of sexual or salivary exchange, or even blood 
transfusion with an infected person.

\textbf{Indirect transmission} occurs when pathogens come in indirect contact with uninfected 
people. This may be in the form of pathogen containing droplets in the air, touching a surface
touched by an infected person, consuming contaminated substances or animals carrying the disease.

There are \textbf{physical barriers} in the body to prevent infection. The skin blocks any 
pathogen's entry into the body. Hairs in the nose trap any such pathogen. \textbf{Chemical barriers}
include mucus secreted by the lining of the respiratory tract in which pathogens become trapped.
Stomach acid kills any pathogen that has been ingested.

\textbf{Mosquitoes} can carry disease. They carry pathogens in themselves which pass into us when
they bite us to suck our blood. They are called vectors\footnote{Vectors are any organisms
that can carry disease}, as a result.

The \textbf{malaria} pathogen is a parasite\footnote{A pathogen is that which depends on another
organism to live}
. When a person infected with malaria is bitten by a
mosquito, it passes into the mosquito's body, where it stays. When that mosquito, carrying the
malarial pathogen bites another uninfected individual, the malaria parasite passes into that person
and they become infected. An infected person suffers high fever, diarrhoea, sweating and 
convulsions. The malaria parasite first enters the bloodstream at the site of the mosquito bite. By
the blood they are transported to the liver, where they mature. Once they have done so, they enter
the bloodstream and infect red blood cells, causing them to burst after a certain period of 
time.

\textbf{Control of malaria} spread can be done by controlling the vectors of the disease, i.e.,
mosquitoes. Use of insecticides, and covering mosquito breeding grounds so that they cannot
reproduce. Draining unnecessary bodies of water and populating those necessary with mosquito
larva eating fish can also be done.

The \textbf{human immunodeficiency virus} (HIV) causes \textbf{auto-immune deficiency syndrome} 
(AIDS). HIV is a viral pathogen and its structure consists of the following:
\begin{itemize}
	\item RNA (genetic material)
	\item Protein coat
	\item Lipid envelope
\end{itemize}
HIV \textbf{spreads} through direct contact. It can only pass by means of direct blood transfusion
and sexual intercourse amongst infected and uninfected individuals. It may pass from mother to
faetus as the virus is small enough to pass through the placenta. Through breast milk, HIV can also
be transmitted.

\textbf{Controlling HIV} comes down to careful screening of blood for the virus before transfusion.
Other than that, limiting sex to one partner and avoiding unprotected sex prevents HIV. Condoms or
femidoms should be used during intercourse. Re-using syringe needles amongst multiple individuals 
should be avoided.

\textbf{Cholera} is a disease caused by a bacterium, which is indirectly transmitted through water 
that has come in contact with the faeces of an infected individual and is drunk by an uninfected
individual. 
To \textbf{prevent} its spread, the sewage system of a country should be well handled
and the waste should be removed carefully. Maintaining good hygiene, i.e. washing regularly
with soap, covering when sneezing and washing hands regularly can lower chances of cholera 
transmission.

\textbf{The action of cholera} consists of the bacterium manifesting in the small intestine. It 
then secretes a toxin which causes the walls of the intestine to release chloride ions. As a 
result, the water potential of the lumen of the small intestine decreases to below that of the
intestine walls (epithelium). Water, as a result, passes from the walls of the intestine into the
lumen, causing the faeces to be watery otherwise known as diarrhoea. Water is lost from the body
in this way, causing the infected individual to be dehydrated. Many ions are also lost in dissolved
form through this water.

\textbf{The treatment of cholera} can be done by rehydrating the infected individual by means of
oral rehydration solution (ORS). 

Excessive \textbf{alcohol consumption} results in reduced self-control and damages the liver as it
is in the liver that the consumed alcohol is broken down. It is drunk as a depressant to obtain
a pleasant numb feeling. Alcohol reduces the reaction times of the drinker, as it slows down nerve
impulse transmission. It is also looked down upon by society.

\textbf{Smokers}, along with tobacco smoke, ingest three other substances: tar, carbon monoxide and
nicotine.

\textbf{Tar} causes bronchitis, in which the goblet cells that line the respiratory tract produce
excessive amounts of mucus. It also increases risk of emphysema, a condition in which alveolar
walls become less elastic, causing them to burst and fuse together. This reduces the surface area
of the gas exchange surface. It is also a carcinogen, meaning it increases risk of cancer.

\textbf{Nicotine} causes the blood vessels to narrow. This may result in coronary heart disease
when the coronary arteries become narrowed.

\textbf{Carbon monoxide} binds to the haemoglobin in red blood cells, rendering them useless and
reducing the body's oxygen carrying capacity. 

It has also been seen that mothers who smoke tend to give birth to babies whose weights are much
less than normal. Mothers are advised not to smoke during pregnancy as doing so increases risks
of miscarriage and oxygen availability for faetus. Babies may be born with withdrawal syndrome
from cigarette smoke and the toxins of the smoke may effect the babies growth as well.

\subsection{Antibiotics}
A \textbf{drug} is any substance taken into the body which alters the body's internal chemical
reactions. 

\textbf{Antibiotics} are drugs which kill bacteria in the body. They are hence used to treat 
bacterial infections. They have no effect on viruses.

\textbf{Antibiotic-resistance} can arise in a population of bacteria in the following situation:
\begin{enumerate}
	\item Person infected with bacteria, takes antibiotics.
	\item Most bacteria are killed, but some mutate a gene which helps them be resistant to 
		that specific antibiotic.
	\item These mutated bacteria are said to be antibiotic resistant.
\end{enumerate}
MRSA is an example of such an antibiotic resistant bacteria. The evolution of antibiotic resistant
bacteria can be reduced by completing antibiotic courses and prescribing them only when necessary.

\subsection{Immunity}
\textbf{Active immunity} is the defense of the body against pathogens by production of antibodies
by lymphocytes in the body itself.

Each type of pathogen has a specifically shaped antigen. The antibody for an antigen is a 
complementarily shaped protein, produced by specific lymphocytes\footnote{A specific lymphocyte
can only produce a specific antibody.}. These antibodies bind to the antigens on the surfaces
of pathogens and either destroy them directly, or join them together so that phagocytosis can be
done more easily.

Active immunity arises once an individual has already been infected by a certain pathogen once, or
by vaccination. 

In vaccination, an individual is injected with weakened pathogens, against which
the immune system defends once the antigens of that pathogen has been detected. When lymphocytes
produce antibodies, they themselves also multiply. Some of these resulting daughter cells are 
memory cells, and they remain in the blood for a long time. When the actual pathogen infects the
individual, these memory cells detect that pathogen's antigens and produce that specific antibody
complementary to that pathogen.

When a population is vaccinated, pathogens have very few places to live, and as a result it becomes
extinct. This is called herd immunity, where even if not everyone is vaccinated the pathogen 
disappears for lack of places to live.

\textbf{Passive immunity} arises when antibodies are taken into the body from another individual.
This is by means of passage of antibodies across the placenta and via breast milk from mother to
child. The body is not being infected by any form of pathogen in this situation, and hence no
memory cells are formed. As a result, passive immunity does not last very long at all. They only
last so long as the antibodies are not broken down in the blood.

In an infant, the immune system is not well developed. That is why antibodies from the mother's
breast milk must be given to the infant, to protect it against infection.

HIV destroys lymphocytes, reducing their number in the blood. Lack of lymphocytes reduces the body's
antibody producing capability, resulting in a weak immune system.

