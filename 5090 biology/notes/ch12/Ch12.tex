\section{Disease and immunity}
\subsection{Disease}
A pathogen is any microorganism that causes disease. A disease that can pass from one host to 
another is called a transmissible disease. This transmission can occur in two modes: directly or
indirectly.

\textbf{Direct transmission} occurs when an infected person comes in direct contact with an 
uninfected person. This may be in the form of sexual or salivary exchange, or even blood 
transfusion with an infected person.

\textbf{Indirect transmission} occurs when pathogens come in indirect contact with uninfected 
people. This may be in the form of pathogen containing droplets in the air, touching a surface
touched by an infected person, consuming contaminated substances or animals carrying the disease.

There are \textbf{physical barriers} in the body to prevent infection. The skin blocks any 
pathogen's entry into the body. Hairs in the nose trap any such pathogen. \textbf{Chemical barriers}
include mucus secreted by the lining of the respiratory tract in which pathogens become trapped.
Stomach acid kills any pathogen that has been ingested.
