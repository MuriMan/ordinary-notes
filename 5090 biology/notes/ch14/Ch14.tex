\section{Coordination and control}
\subsection{Mammalian nervous system}

The nervous system, consisting of the brain, spinal cord and nerves coordinate and control the 
body's functions. The nervous system is in two parts, the peripheral (PNS) and central nervous
systems (CNS).

The CNS consists of the brain and the spinal cord whereas the PNS consists of all the nerves that
branch out from the CNS.

Each nerve consists of thousands of neurones. Neurones are of three types: sensory, relay and
motor. All neurones have three parts, dendrites from which they recieve nerve impulses, the cell
body and the axon through which they transmit nerve impulses. Nerve impulses are electrical in
nature.

Sensory neurones have lon dendrites and a long axon, the motor neurone has shorter dendrites and
long axons and the relay neurone has short dendrites and axon.

Reflex actions are rapid and automatic response to potentially dangerous stimuli, such as touching
a hot surface. We do not think about such actions as the impulses do not reach the brain in such
an action. The following happens in a reflex arc:
\begin{enumerate}
	\item Sensory cells sense something dangerous.
	\item Impulses are created and passed onto sensory neurone.
	\item Sensory neurone pass it onto relay neurones.
	\item Relay neurones pass it onto motor neurones.
	\item Motor neurones stimulate effector cells.
	\item Danger averted.
\end{enumerate}
Reflex actions are effective as no time is wasted in thinking about what to do, which may the 
difference between survival and death in some cases

A synapse is the junction between two neurones. There is a gap between each pair of neurones, the
gap, along with the ending and starting of those neurones is the synapse. The end of the first 
neurone has vesicles with neurotransmitter molecules. When an electrical impulse travels down
this neurone, the vesicles are stimulated to move to the end of the neurone and empty their 
contents. These neurotransmitter molecules diffuse over to and bind to the receptor proteins on 
the dendrites of the
second neurone and an electrical impulse is hence generated in that neurone too.

Synapses ensure that neurotransmitters travel in one direction only as only one side of the synapse 
has receptor proteins and the other has neurotransmitters.

\subsection{Mammalian sense organs}
Sense organs are groups of receptors cells that respond to specific stimuli such as light, sound,
touch, temperature and chemicals. 
The eye is a sense organ consisting of the following parts:
\begin{itemize}
	\item Cornea: Refracts entering light.
	\item Iris: Controls the amount of light entering the eye.
	\item Lens: Focuses light onto the retina.
	\item Ciliary muscles and suspensory ligaments: Control the shape of the lens.
	\item Fovea: Contains the greatest density of light receptors.
	\item Optic nerve: Carries impulses to the brain.
\end{itemize}
Circular and radial muscles are antagonistic. They control the size of the opening of the pupil.
When we are in a bright atmosphere, not much light need enter the eye, so the pupil must be made
smaller. To do so circular muscles contract as radial muscles relax. When light is scarce, the 
opposite occurs.

The light entering the eye is refracted mostly by the cornea, but fine-tuning is done by the lens.
Light coming from far away need not be reflected much, hence the lens must be thin. To do so, 
suspensory ligaments and ciliary muscles are used. The ciliary muscles relax, the suspensory
ligaments become taught and the pull the lens thin. The opposite happens for objects close.

\subsection{Mammalian hormones}
A hormone is a chemical substance, produced by a gland and carried by the blood which alters 
bodily functions, that of some specific organs. A few glands and their hormones are:
\begin{itemize}
	\item Adrenal glands: Adrenaline
	\item Pancreas: Glucagon and insulin
	\item Pituitary gland: FSH, LH, (see Section 16)
	\item Testes: Testosterone
	\item Ovaries: Oestrogen and progesterone.
\end{itemize}

Adrenaline is produced in life threatening situations, it stimulates the release of glucagon
which results in glycogen stored in the liver to be released into the blood as glucose so that 
energy can be used. It increases heart rate to provide muscles with oxygen and glucose as much as
they require. Breathing rate increases too, to absorb as much oxygen as possible.

Nervous and hormonal control differ in that nervous control is faster yet last longer but hormonal
control is slower yet lasts as long as the hormones are not broken down.

\subsection{Homeostasis}
Homeostasis is the maintenance of a constant internal environment. There are set points of 
conditions in the body, when the body goes beyond or below which, negative feedback occurs and the
body is stimulate to undo this change.

The skin consists of hairs, muscles to erect those hairs, sweat glands and ducts, receptors, 
sensory neurones, blood vessels and fatty tissue.

Insulation through adipose tissues under the skin prevent the loss of heat from the body and
entrance of heat from outside the body.

Hypothalamus

Insulation through adipose tissues under the skin prevent the loss of heat from the body and
entrance of heat from outside the body.

The hypothalamus has receptors which can sense temperature of blood passing through it. When the
temperature of blood is above or below the set point of body temperature the hypothalamus 
stimulates actions to bring it back to set point.

Sweating happens when we are too hot. When sweat evaporates from our body, the body gets cooler.

Shivering happens when we are too cold, the movement of muscles along with the respiration required
to cause it warms us up.

When we are cold our hairs stand up so as to trap warm air.

Vasodilation occurs when we are too hot and heat must be lost to the atmosphere from the blood
vasoconstriction is the opposite and occurs when we are too cold and heat must be conserved in the
blood. Vasodilation is the dilation of arterioles near the skin surface and vasoconstriction is
the constriction of those arterioles.

\subsection{Blood glucose control}
Cells need steady amount of glucose supply to respire and carry out their processes. Too much 
glucose can be mad as they can mess with water potentials causing water to move out of the cells.

Insulin is the hormone that when secreted, stimulates liver to convert more and more glucose to
glycogen. Glucagon does the opposite in that it stimulates the liver to convert more glycogen to
glucose.

Type 1 diabetes is a condition where the body cannot produce its own insulin resulting in high
blood glucose concentrations and high urine glucose concentrations. Such persons must eat 
controlled amounts of carbohydrates as eating too much may make them feel unwell and eating too 
little will result in them being lethargic as they have no stored glycogen. Their treatment consists
of injecting insulin directly into their blood stream.

