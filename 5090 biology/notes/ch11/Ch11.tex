\section{Transport in humans}
\subsection{Circulatory system}

A \textbf{circulatory system} consists of a network of blood vessels, a pump or heart to keep the
blood flowing and valves which ensure the flow is in one direction only.

A \textbf{double circulatory system} is that in which blood passes through the heart twice for
a complete circulation. In such a system, blood at low pressure is sent to the lungs, and its
pressure is raised once again after it arrives at the heart and is sent off to the body.

\subsection{Heart}
The \textbf{mammalian heart} is essentially a muscle with four chambers. To prevent backflow,
valves are present at specific locations around the heart. 
The thickness of the muscles of each chambers varies depending on the distance they must pump
the blood. The atria are significantly less thick than ventricles as they need only pump blood to 
the ventricles which are right beneath them. 
The left\footnote{Note that the left and right sides of the heart is reversed, as we
consider the right to be the right side of whose heart we are observing.} ventricle is 
significantly thicker than the right, as the left pumps blood to the whole body whereas the right
only pumps to the lungs.

The heart is composed of two sides, the left and the right, separated by the septum. On the top of 
each side is an atrium
and on the bottom, a ventricle. Seperating the two are atrio-ventricular valves.
\textbf{The functioning of the heart} consists of the following steps:
\begin{enumerate}
	\item Deoxygenated blood from the body flows into the right atrium, simultaneously oxygenated
		blood from the lungs flow into the left atrium. At this moment, the atrio-ventricular
		valves on both sides are closed.
	\item The atria contract, pressuring blood through the atrio-ventricular valves into their
		respective ventricles. The valves ensure blood does not flow back into the atria.
	\item The ventricles contract, the atrio-ventricular valves stay closed, the semi-lunar valves
		that are at the entrance of the aorta and pulmonary artery of the left and right heart
		respectively open to allow blood to flow through those blood vessels\footnote{
		Note that, blood is pumped away from the heart in blood vessels called arteries and toward the 
		heart in those called veins.}
\end{enumerate}
Note that, the atrio-ventricular valve of the left side of the heart is called the bicuspid valve
and that on the right is called the tricuspid valve.

The above events occur multiple times every minute, and the number of times it occurs is called the
\textbf{heart rate}. It can be measured in the following ways:
\begin{itemize}
	\item The heart makes a lub-dub sound when it beats, caused by the opening and closing of the
		valves. Counting the number of times lub-dub is heard per unit time can give heart rate.
	\item Pulse rate can be measured by placing one's hand on any artery, and counting the number
		of times it pulses per unit time.
	\item An echocardiograph (ECG) can be used to measure and record electrical activity in the
		heart using electrodes stuck into the person's body.
\end{itemize}
\textbf{Coronary heart disease} (CHD) occurs when the arteries that supply the heart itself with blood,
called the coronary arteries, become blocked in some way. It can occur due to the following 
factors:
\begin{itemize}
	\item Diet: Eating high amounts of saturated fats results in high cholesterol concentration.
	\item Sedentary lifestyle: Those with lethargic lifestyles tend to run a higher risk to develop
		CHD. 
	\item Stress: Leading a high stress lifestyle may also result in CHD.
	\item Smoking: Smoking greatly increases risk of CHD.
	\item Genetic predisposition: Many carry hereditary genes which make them susceptible to CHD,
		such an individual must lead a healthy lifestyle so as to avoid suffering from the disease.
	\item Age and gender: Older people run a greater risk of contracting CHD, and men are more 
		prone to CHD than women.
\end{itemize}
Regular exercise and having a healthy diet is the key to combat CHD. Regular exercise keeps the 
mind and body fit, keeps the blood pressure at a good value, reduces chance of excessive weight 
gain. They make one feel fit. Other than that, avoiding fatty, greasy, animal-based foods is the
key to preventing CHD as such foods contain saturated fats which increase cholesterol levels in the
body leading to CHD.

\subsection{Blood vessels}
It is through the aorta that oxygenated blood flows out of the heart and to the rest of the body parts. 
Through the inferior and superior vena cava, deoxygenated blood enters the right side of the heart.
The hepatic vein and artery carry blood to and from the heart and liver, respectively. The hepatic
portal vein is a blood vessel going from the small intestine to the liver, carrying nutrients to
be assimilated in the liver. The renal artery and vein carry blood to and from the heart and the
kidneys, respectively.

\textbf{Arteries} carry blood away from the heart. They mostly contain oxygenated blood, except the
pulmonary artery. They have thick, muscular walls with elastic tissue, that bounce back with the
flow of blood to allow the blood to flow smoothly. The lumen of an artery is relatively small, but
not as small as capillaries.

\textbf{Veins} carry blood to the heart. They mostly contain deoxygenated blood, except the 
pulmonary vein. They have thin walls, also composed of muscles and elastic fibres. They have a 
relatively large lumen and have valves to prevent backflow of blood. Veins do not carry 
particularly high pressure blood, so, such as to not slow down speed of blood flow they have a 
large lumen. They need not thick walls as they carry not high pressure blood.

\textbf{Capillaries} are very thin blood vessels which surround cells, they are the middle ground
between an artery and a vein. They have very small lumens and walls only one cell thick. It is 
through capillaries nutrients, oxygen, etc. diffuse in and out of the blood and cells. They are 
small so as to penetrate every part of the body and supply it with blood as it requires.

\subsection{Blood}
Blood is a mixture consisting of plasma, red and white blood cells, and platelets. Red blood cells
are biconcave and lack all cell organelle. White blood cells are of two types, lymphocytes and
phagocytes. Phagocytes have lobed nuclei, whereas lymphocytes have nuclei that take up almost the
whole cell. Platelets are tiny structures, smaller than both the blood cells.

\textbf{Red blood cells} contain haemoglobin, which is a substance to which oxygen binds to form
oxyhaemoglobin. It is in this form that oxygen is transported around the body. Red blood cells
lack any cell organelle to maximise space for oxyhaemoglobin. They are biconcave in shape to 
maximise surface area for diffusion of oxygen.

\textbf{White blood cells} are involved in immunity of the body. Lymphocytes are cells that produce
antibodies to kill any and all pathogens and phagocytes engulf those pathogens to kill them.

\textbf{Platelets} are involved in blood clotting. When a blood vessel is broken, the platelets
release a substance to convert soluble fibrinogen present in blood to fibrin, an insoluble protein
that forms a mesh-like structure around the wound. Blood cells get trapped in this mesh, preventing
excessive blood loss.

\textbf{Plasma} is a substance consisting of mostly water, in which many substances are dissolved
for transport. Such as: glucose, amino acids, mineral ions, hormones, carbon dioxide, urea, vitamins
and plasma proteins.

Oxygenated blood flows in from arteries into capillaries. Here, oxygen diffuses out from inside
red blood cells, into cells and carbon dioxide diffuses into the plasma. Such exchanges occur
before the branched capillaries congregate into veins carrying deoxygenated blood. Plasma leaks
out from capillaries during this exchange. This leaked plasma is called tissue fluid and it helps
in diffusion of substances.
