\section{Human gas exchange}
\subsection{Human gas exchange}
The human gas exchange surface consists of the lungs. They are specialised for exchange of gases
by having:
\begin{itemize}
	\item Large surface area, to maximise diffusion.
	\item Thin surface distance; the surface is only one cell thick so as to minimise distance for
		diffusion, maximising diffusion rate.
	\item They are well supplied with capillaries so as to diffuse as much oxygen into the blood
		and as much carbon dioxide out of the blood as possible. The lungs take in sufficient air
		so as to cause maximum diffusion as well.
\end{itemize}
In the atmosphere, \textbf{the proportions of gases} present follow: 21\% of oxygen $\textrm{O}_2$
, 0.04\% 
of carbon dioxide ($\textrm{C}_2\textrm{O}$). These are the gas proportions that we breathe in,
i.e., that we inspire. The air that we breathe out, exhale or expire consists of the following
compositing: 16\% oxygen, 4\% carbon dioxide and very high amount of water vapour, compared to that
inhaled.

\textbf{The respiratory tract} consits of the larynx, trachea, bronchus and bronchioles.
Once air has entered through our noses, it passes the larynx or the voice box. It then goes through
the trachea. The trachea splits into two bronchi (bronchus, singular) which branch out into 
bronchioles further into
grape like structures called alveoli. It is these alveoli through which gaseous exchange in humans
occur. These are balloon-like in that they blow up when we inhale, and deflate when we exhale.

\textbf{Inside these alveoli}, gases accumulate when we breathe in, and around these there are 
capillaries
enveloping it. The alveolar wall is only one cell thick, and is moist, so as to minimise the 
distance oxygen has to travel to diffuse into the blood in the surrounding capillaries. The surface
area in contact with the blood vessels is also large so as to maximise diffusion. They are 
constantly supplied with blood so that blood can be oxygenated. They also have a good supply of 
oxygen from our constant breathing in and out, also called ventilation.

The ribs are the cage-like bones surrounding our lungs and heart. That which is enveloped by the 
ribs is called the thorax. In between and around each rib-bone there are intercostal muscles, which
are arranged antagonistically\footnote{A pair of muscles where, when one contracts the other 
relaxes and vice versa.}, namely the external and internal intercostal muscles. Beneath the ribs,
there is the diaphragm, which is a muscle separating the thorax, i.e. the chest cavity from the
abdominal cavity.

\textbf{Breathng in} occurs when the pressure inside the thorax, i.e. the lungs lowers to below
that of atmospheric pressure. Air from outside the body rushes in through the nose and mouth into
the alveoli when this happens. To do this, the volume of the thorax must be maximised, which is 
done by contraction of the diaphragm from its relaxed dome shape. Alongside that, the external
intercostal muscles contract, relaxing the internal intercostal muscles. This results in the ribs
moving up and outwards, decreasing internal pressure and causing air to rush into the lungs.

To \textbf{breathe out} the exact opposite must occur. The thorax's volume must lower, so as to
increase lung pressure to be greater than that of the surrounding atmosphere. To do so, the thorax
relaxes into its dome shape and the internal intercostal muscles relax whereas the external 
intercostal muscles relax, causing the ribs to move down and in. As a result, air rushes out of the
lungs into the atmosphere.

\textbf{Physical activity has effects} on the rate and depth of breathing. When performing physical
tasks, such as running, our body needs lots of oxygen as fast as possible to respire aerobically,
to produce energy for muscle contraction. This is why we breathe faster, at a higher rate, and take
deeper breaths, to maximise oxygen in our blood.
Often, this oxygen demand is not met, and the body has to
compensate by performing anaerobic respiration, which releases less energy in comparison to aerobic
respiration. Anaerobic respiration produces lactic acid, which accumulates in the blood and 
muscles. To break down this lactic acid, we continue breathing in quicker and deeper even after we
have completed the physical task, as the breakdown of lactic acid requires oxygen. This is known 
as \textbf{paying off the oxygen debt}, this breakdown occurs in the liver and we can
say we borrowed oxygen when we performed anaerobic respiration, and we are paying it off by taking
in the oxygen we need. The rate and depth of breathing is controlled by the brain, which, when it
senses that the pH of blood has lowered due to lactic acid, stimulates the diaphragm and 
intercostal muscles to contract and relax harder and more often. Faster and deeper breaths result.

The trachea, bronchi, even the inside of the nose is lined with \textbf{goblet cells and cilia}.
Goblet cells produce mucus, a chemical which traps any pathogen or particle that enters the 
respiratory tract. Cilia are hair like structures which continuously beat upward, pushing mucus
upward toward the top of the throat to be swallowed. Once they are, the acid in the stomach 
destroys any possibly harmful pathogen that may have been stuck in that mucus.
