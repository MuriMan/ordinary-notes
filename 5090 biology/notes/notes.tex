\documentclass[twocolumn]{article}

\begin{document}

\section{Cells}
\subsection{Cell structure and function}
An animal cell is composed of the following organelles:
\begin{enumerate}
	\item Cell membrane: A \textbf{partially permeable} membrane that lets certain substances 
		but not others. It seperates the contents of the cell from its surroundings. 
	\item Cytoplasm: A clear jelly-like substance, composed of mainly water and dissolved
		substancese (proteins). \textbf{Metabolic reactions}, which are chemical reactions
		of life take place in the cytoplasm.
	\item Nucleus: The site of storage of genetic information, kept on 
		\textbf{chromosomes} which are made of \textbf{DNA} (deoxyribonucleic acid).
	\item Mitochondria: Mitochondria are the site of \textbf{aerobic respiration}, the 
		process through which energy is released from glucose.
	\item Ribosomes: The site of protein manufacture. They do so by reading the 
		instructions off of DNA, linking chains of amino acids, forming proteins.
	\item Vesicles or small vacuoles: Small membrane bound organelle containing certain
		solutions.
\end{enumerate}

A plant cell is composed of the following organelle:
\begin{enumerate}
	\item Cell membrane: A \textbf{partially permeable} membrane that lets certain substances 
		but not others. It seperates the contents of the cell from its surroundings. 
	\item Cell wall: A stiff protective fibrous wall, made of \textbf{cellulose}. It is
		\textbf{fully permeable}.
	\item Cytoplasm: A clear jelly-like substance, composed of mainly water and dissolved
		substancese (proteins). \textbf{Metabolic reactions}, which are chemical reactions
		of life take place in the cytoplasm.
	\item Nucleus: The site of storage of genetic information, kept on 
		\textbf{chromosomes} which are made of \textbf{DNA} (deoxyribonucleic acid).
	\item Chloroplast: Membrane-bound organelle consisting of green coloured pigment 
		\textbf{chlorophyll}. Chlorophyll absorbs energy from sunlight, which is then
		used for photosynthesis. They often contain starch grains.
	\item Mitochondria: Mitochondria are the site of \textbf{aerobic respiration}, the 
		process through which energy is released from glucose.
	\item Ribosome: The site of protein manufacture. They do so by reading the 
		instructions off of DNA, linking chains of amino acids, forming proteins.
	\item Vacuoles: A large \textbf{cell sap} filled membrane bound organelle.
\end{enumerate}

\subsection{Specialised cells, tissues and organs}
Large organisms need to perform specific functions, to do so they need specific cells,
such cells are called \textbf{specialised cells}.

Examples and details are listed:
\begin{itemize}
	\item Ciliated cell: These line the trachea and the bronchci. They continuously beat 
		upward to push up bacteria or dust particles that becocme trapped in them to prevent
		lung blockage.
	\item Neurone:  They are part of the nervous system. These are of three types: 
		sensory, relay and motor. Each is composed of two fibrous parts: dendrites 
		(attached to the cell body) and the axon. Dendrites pick up nerve impulses, 
		electrical signals from nearby neurones, which is passed along the cell body and 
		the axon to possibly another neurone.
	\item Red blood cell: These are found in the mammalian blood. They consist of the red
		pigmented susbtanes \textbf{haemoglobin} which itself is a protein containing 
		iron. Haemoglobin combines with oxygen to form \textbf{oxyhaemoglobin}. It is
		through oxyhaemoglobin that oxygen is transported across the body. They lack
		nuclei and mitochondria, this is so to maximise space oxygen storage. They have
		a \textbf{biconcave} shape, to maximise surface area to volume ratio. They are
		small in size so that they can squeeze easily through capillaries. They are 
		produced in very large numbers to maximise oxygen transport capabilities of the
		body.
	\item Sperm cell: The male human gamete. It is composed of three parts.
	
\end{itemize}

\end{document}
