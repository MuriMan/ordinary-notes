\section{Respiration}
\subsection{Respiration}
\textbf{Respiration} is the chemical process by which energy is released from glucose in all living
cells. This energy is required for many processes, including:
\begin{itemize}
	\item Muscle contraction
	\item Protein synthesis
	\item Cell division
	\item Active transport
	\item Growth
	\item Passage of electrical nerve impulses
	\item Maintenance of a constant body temperature (homeostasis).
\end{itemize}

\subsection{Aerobic respiration}
\textbf{Aerobic respiration} is the breaking of glucose using oxygen to release energy. 
Given below, are the word and chemical equations denoting the process of respiration:
\begin{center}
	glucose + oxygen $\rightarrow$ carbon dioxide + water

	C$_6$H$_{12}$O$_6$ + 6O$_2$ $\rightarrow$
	6H$_2$O + 6CO$_2$
\end{center}

\subsection{Anaerobic respiration}
\textbf{Anaerobic respiration} is the breaking of glucose without oxygen to release a relatively
little amount of energy. In humans, the breakdown occurs as follows:
\begin{center}
	glucose $\rightarrow$ lactic acid
\end{center}
In yeast, the following happens:
\begin{center}
	glucose $\rightarrow$ alcohol + carbon dioxide
\end{center}
The parts concerning Excess Post-exercise Oxygen Consumption (EPOC), otherwise known as paying of
the oxygen debt is given in the last part of Section 9.

During exercise, the heart rate is fast so as to supply oxygen to all the parts that need it as
quickly as possible. After exercise, the heart rate remains fast for some time to transport lactic
acid to the liver where it can be broken down using oxygen.
