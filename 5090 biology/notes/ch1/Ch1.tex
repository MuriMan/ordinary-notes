\section{Cells}
\subsection{Cell structure and function}

A cell is the smallest unit from which all organisms are made. These structures are so small
that microscopes must be used to observe them. Microscopes are equipment that can magnify objects
many times, meaning even the tiniest objects will appear large under a microscope. In case of cells
and such, methylene blue or iodine solution must be used to stain the cells to increase their
visibility under the microscope.

Most animal cells consist of the following: cell membrane, mitochondria, nucleus, cytoplasm, and
ribosomes. 

Cell membranes  are partially permeable in that not all substances will be allowed to 
pass through the membrane. It is a thin layer consisting of mostly protein and fat. It is what
separates the contents of the cell from those in the surroundings. 

Mitochondria are organelle which are so small that they can only be clearly seen under an electron
microscope. This is the site of aerobic respiration inside the cell. Aerobic respiration is the
process by which glucose is broken down to release energy, which is then used for the cell's
metabolic reactions\footnote{Metabolic reactions are those which occur inside living organisms.}.

The nucleus of a cell is where the genetic information of the cell is stored. It is stored in
the form of chromosomes, which are structures made from deoxyribonucleic acid\footnote{Candidates
need not remember this name.}, DNA. Chromosomes are long and thin structures which are so small 
that they are only clearly visible when a cell is dividing.

The cytoplasm is a clear jelly consisting of mostly water. It has many substances dissolved in it,
especially proteins. All metabolic reactions take place in the cytoplasm. Ribosomes are structures only visible under an electron microscope. This is the site of protein
synthesis, wherein instructions in the DNA of the chromosomes are used to link together long chains
of amino-acids.

Vesicles are temporary membrane bound organelle that contain solutions inside animal cells.

All of the above organelle are also present in plant cells, with the exception of the vesicle and
some additional organelle, specifically: chloroplasts, sap vacuole, and, cellulose cell wall.

Chloroplasts are membrane-bound organelle that contain the green pigment called chlorphyll. They
often contain starch grains. Chlorophyll is responsible for absorbing the energy from sunlight, 
which is then used to make food by photosynthesis.

The sap vacuole is another membrane bound organelle filled by a fluid, which is cell sap in plants.
The pressure in the vacuole due to it being full presses outward on the cell itself, helping it
keep its shape.

The cellulose cell wall is a fully permeable wall surrounding the cell in plants, and is made of
cellulose, which is a polysaccharide (see Section 4). These protect and support the cell, and when
excess water has entered the cell, it is stopped from bursting by the cell wall.

Bacteria are prokaryotic cells (see Section 2), and hence have no proper nucleus and only have 
their genetic material freely suspended in their cell cytoplasm in circular form. They often 
contain additional rings of DNA as plasmids. Bacterial cells also have ribosomes, yet lack 
mitochondria. The cell wall in bacterial cells is composed of peptidoglycan, they may have flagella,
which are structures that allow mobility to the cells. They may have a further protective capsule
shaped layer, called the capsid.

\subsection{Specialised cells, tissues and organs}
Some cells can become specialised and their structures are related to their speciifc functions,
examples of such cells will be covered.

A tissue is a group of similar cells that work together to perform a particular function. An organ
is a group of tissues that work together to perform a particular function. An organ system is made
up of sesveral organs to perform a particular function. Multiple organ systems make up an organism,
which is a living thing.

Note that, when looking at magnified images, the following formula applies:
$$
\textrm{magnification} = \frac{\textrm{image size}}{\textrm{actual size}}
$$
