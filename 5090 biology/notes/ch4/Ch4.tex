\section{Biological molecules}
\subsection{Biological molecules}

Carbohydrates are biological molecules containing carbon hydrogen and oxygen. Lipids are those
which consist of the same elements, in different proportions. Proteins are composed of carbon,
hydrogen, oxygen and nitrogen. DNA contains phosphorous, carbon, hydrogen and oxygen.

All of the above are large molecules, or polymers, which are made from smaller molecules, or 
monomers. The monomer for carbohydrates such as starch, cellulose and glycogen is glucose. Proteins
are made from amino acids. Lipids are made from fatty acids and glycerol and DNA is made from
nucleotides.

Starch is tested for by addition of iodine solution, which will turn blue black from brown in 
presence of
starch. Glucose and maltose are reducing sugars, which, when heated with Benedict's solution, turn
green, yellow or orange-red from blue depending on the concentration of the sugar present. 
Proteins are tested
for by addition of biuret reagent, which changes to violet from blue in presence of proteins.

Lipids are tested for by first mixing the sample in ethanol, in which the lipid dissolves. This
ethanol is then put into water, where, if tiny droplets are formed giving the solution a milky
appearance, fats are present. Such droplets floating in water is called an emulsion, and the
test is named accordingly.

