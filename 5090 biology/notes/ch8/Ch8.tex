\section{Human nutrition}
\subsection{Diet}

The diet of a person is the food eaten in a day. A balanced diet is that which contains all
required nutrients in suitable proportions, which provides the right amount of energy.
There are seven types of nutrients that a human requires in their diet for it to be balanced.

Carbohydrates are needed for energy, examples include starch and sugar. Most countries have a 
staple food that tends to be carbohydrate rich such as rice, corn, wheat etc.

Lipids are needed for energy and also to make cell membranes. Excess fats and oils\footnote{A solid
lipid is called a fat, a liquid lipid is called an oil} is stored under the skin as adipose tissue.
In this form, it insulates the body by reducing heat lost to atmosphere. It may form layers around
organs, to provide mechanical protection. Cooking oils, meat, egs, dairy products and oily fish 
contain such lipids.

Proteins are required to make new cells, i.e., for growth. Haemoglobin, insulin and antibodies
are made from insuline. Meat, fish, eggs, dairy products, peas and beans, nuts and seeds contain
protein.

Vitamins are organic substances which are required in miniscule amounts. Lack of such vitamins
may cause deficiency diseases.

Lack of vitamin C causes scurvy, which causes muscle and joint pain. They cause gums and other
places to blead. Vitamin C is found in citrus fruits.

Vitamin D is found in butter, egg yolk and exposure to sunlight is stimulates the skin to make
it. Calcium absorption depends on presence of vitamin D. Lack of this vitamin causes rickets,
a disease in which bones become soft and deformed.

Minerals are inorganic substances of which we require very little.

Calcium is a mineral found in milk, bread and dairy products. It is needed to make bones, teeth
and for blood clotting. Deficiency of calcium causes brittle bones and teeth and poor blood 
clotting. Lack of calcium may also cause rickets.

Liver, red meat, egg yolk and dark green vegetables contain iron. This mineral is required to make
haemoglobin\footnote{Any word starting with ``haem-" tends to have something to do with iron}, the
red pigment in blood which is responsible for carrying oxygen. Lack of iron in the diet causes
anaemia, in which there are not enough red blood cells so the tissues do not get enough oxygen
delivered to them.

Fibre, also called roughage, is needed to have the alimentary canal working properly. Peristalsis
is the rhythmical muscle contractions that move food down the alimentary canal. Harder foods 
stimulate these muscles more when compared to softer foods. Fibre hence is required to have this
system in good working order and helps to prevent constipation. Fibre consists of undigestable 
matter, such as cellulose from plant cell walls. This is why all plants contain some amount of
roughage. Cereal grains, wholemeal bread and brown rice are all sources of roughage.


More than 60\% of the human body is water. Cell cytoplasm is mostly water. Metabolic reactions
occur in solution with water as the solvent. A large part of blood is plasma, which is also mostly
water, in which substances are dissolved before being transported. Enzymes and nutrients are
digested in solution. Urine is also mostly waste product. Drinking fluids and fruit give us the
water that we need.
\subsection{Human digestive system}

The digestive system consists of the mouth, salivary glands, oesophagus, stomach, the small
intestine (duodenum and ileum), pancreas, liver, gallbladder and large intestine (colon, rectum
and anus).

Eaten food consists of large pieces, which are made smaller by means of physical digestion, where
no change happens to the chemical content of these large pieces. This is done to increase surface
area of food for enzyme action in chemical digestion.
These small pieces are broken down further into small molecules, called chemical digestions. These 
small
molecules can then be transported through the blood and later be assimilated into the body.

Digestion begins in the mouth, where teeth break large food pieces into smaller pieces. Teeth are
of different types.

Incisors are sharp edged, chisel shaped teeth present at the front of the mouth which are used to
bite off food. Canines are more pointed teeth at either side of these incisors which are used to
tear off food. Premolars and molars are large flat shaped teeth used for chewing food. There are
some teeth that grow around age 18~21. called wisdom teeth, these grow at the back of the jaw
and are molars.

Each tooth is embedded into the gum, which is embedded into the jawbone. The exposed part of the
tooth is made of enamel, which is the hardest substance the human body produces. However acids
can dissolve it. Beneath this enamel layer is that of dentine. Dentine is a bone like structure
which has chennels containing living cytoplasm. The centre of the tooth contains nerves and blood
vessels, which supply the cytoplasm in the dentine with nutrients and oxygen. The part of the tooth
which is in the gum, i.e., is unexposed is surrounded by a layer called cement. It has fibres
growing out of it which attach the tooth to the jawbone but allow it to move slightly which biting
or chewing.

The alimentary canal is essentially a long, winding tube running from the mouth to the anus. Some
parts are kept separated from others, where necessary, by sphincter muscles. Each part of the
alimentary canal is kept lubricated with mucus secreted by goblet cells lining the wall of the
alimentary canal.

In the mouth, saliva is a fluid secreted by salivary glands. It consists of water, mucus and the
salivary amylase enzyme which is responsible for chemical digeston of starch. The mucus in the
saliva helps to bind the chewed food into a small ball, called the bolus, which allows it to slide 
easily down the oesophegus when swallowed. Here, amylase begins to digest starch to maltose. The
oesophagus consists of antagonistically arranged longitudinal and circular muscles, whose rhythmic 
movements help food down
the alimentary canal, this rhythmic movement is called peristalsis. The food then reaches the
sphincter muscle at the entrance of the stomach. When this muscle relaxes, the bolus enters
the stomach. The stomach has a strong muscular walls which contract and relax to mix the food
with enzymes and mucus. The walls of the stomach contain goblets cells which secrete mucus, enzymes
and hydrochloric acids. The churning of the stomach breaks the bolus down into further smaller
pieces. The enzymes secreted are proteases which break proteins in the bolus to poly peptides and
then to amino acids.

When the sphincter muscle at the end of the stomach is relaxed, the bolus, now turned into chyme,
enters the duodenum. Here, bile, which is a fluid mixture with a green colour and an alkaline
nature, is mixed with the chyme, neutralising the acidity of it which resulted from the stomach
acid. The bile enters the duodenum from the bile duct from where it was stored in the gallbladder.
Bile is produced by the liver. Bile also emulsifies the lipids in the chyme, increasing their
surface area for enzyme action. A tube called the pancreatic duct leads from the pancreas into the
duodenum, through which pancreatic juice flows into the duodenum. The pancreatic juice contains
enzymes such as amylase, maltase, lipase and protease. Pancreatic amylase breaks down starch to
maltose, which maltase breaks to glucose. Lipases act on the emulsified lipids and break them down
to fatty acids and glycerol. Proteases act on proteins to break them down to amino acids. Once
food has passed the duodenum, it enters the ileum. The initially ingested food now consists of 
small soluble molecules. These molecules must now pass through the walls of the small intestine
to be absorbed into the blood, this process is hence called absorption. The inner wall of the
duodenum is covered with tiny projections called villi, of which there are many foldings forming
microvilli. On the membranes of these cellular structures enzymes such as maltase acts to breakdown
maltose to glucose. These monomer units; glucose, fatty acids, glycerol, amino acids and mineral
ions, are absorbed through these microvilli. Most of these substances pass into the blood in the
capillaries in the villi. All these capillaries join into the hepatic portal vein through which
blood transports all these substances to the liver. Fatty acids and glycerol pass into the lacteal
part of villi, which eventually empty into the blood. The villi are shaped with such microvilli
to maximise surface area.

Note that protease consists of specifically two proteins, pepsin and trypsin which are both 
involved in the same task.

Hydrochloric acid in the stomach is present to kill any accidentally ingested bacteria.

The protease in the stomach has an an acidic optimum pH whereas those in the ileum have a neutral
to alkaline optimum pH.

\subsection{Absorption and assimilation}
It is in the small intestine, that all nutrients are absorbed. Absorption is the movement of 
nutrients from from the intestine to the cells lining the digestive system and into the blood by
diffusion, osmosis and active transport.

Assimilation is the uptake and use by cells of nutrients from the blood.

Villus are folded in a way to form many many microvilli to maximise surface area of the small
intestine to maximise movement of nutrients and rate of movement of these nutrients from the
intestinal lumen into the capillaries of the villi. Most of the water ingested is absorbed in the
small intestine, also into the blood, the colon also does the same. The ingested food that remains
after action of the small intestine, travels into the colon, where remaining water is absorbed.
The final undigested part of food is stored into the rectum until the sphincter muscle to the
anus is opened, and the undigested part is egested.

Most molecules and ions travel through the hepatic portal vein to the liver to be absorbed.
