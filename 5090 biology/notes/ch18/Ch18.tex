\section{Biotechnology and genetic modification}
\subsection{Biotechnology}

Biotechnology is the usage of organisms to produce required substances. Yeast is such an organism
which can be used to produce ethanol and carbon dioxide, as it produces them when it undergoes
anaerobic respiration:
\begin{center}
	\ce{glucose ->[absence of oxygen] ethanol + carbon dioxide}
\end{center}
The above reaction is useful to produce alcoholic beverages, and in baking, as bread can be made
to rise by the carbon dioxide produced when yeast is kept in the dough, where it anaerobically
respires the carbohydrates in the dough. Note that for yeast's anaerobic respiration, a source
of glucose is always required.

Bacteria are useful in biotechnology and genetic modification as there are no ehtical concerns over
their manipulation and growth and the presence of plasmids.

Fermenters can be used for the large-scale production of useful products by bacteria and fungi,
by controlling temperature, pH, oxygen, nutrient supply and waste products.

Before a fermenter is used, it is sterilised to remove any and all microorganisms so that they do
not contaminate the culture (population of microorganisms growing in a nutrient liquid or on agar
jelly).

Nutrients such as glucose and nitrates may be added to the fermenter for the microorganism, to
provide energy and raw material for production of amino acids and subsequently proteins.

Aerobic respiration of these organisms will depend on the presence of oxygen, for which air can be
bubbled through the fermenter. Waste gasses are made to escape through an outlet, so that no 
pressure can build up inside the fermenter.

The contents are constantly mixed using a paddle attached to a motor. This makes sure the every
organism is well supplied with nutrients and oxygen.

Temperature is maintained so that enzymes in the microorganism can work effectively, to produce
the required product. Heat is released as microorganisms respire, to counteract which, cold water
is passed around the fermenter.

The pH of the mixture inside the fermenter is constantly monitored and controlled by adding small
amounts of acidic or alkaline substances when it goes above or below the set point. It is important
to keep the pH constant to keep them at optimum for the microorganisms enzymes.

Enzymes can be used too. Pectinase is an enzyme which extracts fruit juice. Pectin is a substance
which helps to stick plant cells together, which pectinase breaks down. This makes it easier
to squeeze down the fruit and produces a clearer juice. Cellulase can also be used to digest
cellulose, adding to the clearness of the juice.

Blood or food stains contain biological substances from animals, which can be broken down easily
by enzymes. That is why enzymes are added to biological washing powders, so that they can catalyse
the breakdown of these substances thereby removing those stains.

All human babies can digest lactose, a sugar found in milk. As we grow, we stop producing the
enzyme required to digest this sugar. Such individuals are called lactose intolerant and suffer
discomfort when they ingest lactose. Milk can be treated with lactase, the enzyme that catalyses
the breakdown of lactose, so that lactose intolerant individuals can enjoy milk and its benefits.

\subsection{Genetic modification}

Genetic modification is changing the genetic material of an organism by removing, changing or 
inserting individual genes.

The gene controlling insulin production in humans can be inserted into the plasmids of bacteria,
by use of restriction enzymes which cut DNA in particular places, maintaining sticky ends and 
inserting required gene into the bacterial plasmid, making it into a recombinant plasmid which can
then be reinserted into the bacteria making them produce insulin. The plasmid here, is called a
vector. The bacteria will be allowed to reproduce and produce the insulin in a fermenter in optimum
conditions, maintenance of which has been describe above.

The same process can be used in crops, only their DNA will be directly modified, no vectors will
be used, to confer resistance to herbicides, pests and to make them more nutritionally valuable
to provide additional vitamins.

Such genetically modified (GM) plants can be expensive for farmers, especially when producing 
companies restrict farmers from keeping the seeds for next season and forcing them to buy each
season. Producing vitamin rich crops diverts the real cause of the problem, which is a wrong diet.
Herbicide resistant crops save labour costs and herbicide costs for farmers, and they cause no
damage to neighbouring plants. However, this herbicide resistant gene can pass into weeds that
grow very close to them, producing superweeds that herbicides can no longer kill. This matter is
very unlikely nonetheless. Many think pest resistant crops are harmful to people too, no scientific
evidence has risen yet about the matter. Organisms that do not harm crops may be harmed by the
pest resistant crops, causing ecological damage, yet no studies have comprehensively shown evidence
of this matter. However, pests may evolve to be resistant to the crops' pest resistance mechanism,
something which has occured before. A solution is to grow resistant crops coincident to the non
resistant crops. Insects will then feed on the non resistant ones and will hence have no need to
be resistant to the pest resistance crop. 
