\section{The internet and its uses}
\subsection{The internet and world wide web}

The internet is upon which the world wide web is based. In other words, the internet is the 
infrastructure, on which stands the collection of websites and webpages which make up the world
wide web.

In short, the internet is a very large global network which allows users access to the world wide
web. The world wide web is the collections of all websites and webpages available to the public.

Each website and webpage in the world wide web has a 
unique text based address called its uniform resource
allocator (URL). A URL consists of three main components: the protocol, domain name and webpage or
file name. It is arranged into a format resembling what follows:

\begin{center}
\begin{BVerbatim}
protocol://www.domain-name/webpage
\end{BVerbatim}
\end{center}
The way users obtain webpages from web servers is based on the hypertext transfer protocol (HTTP).
It consists of users requesting the domain name server (DNS) for a certain website, and the web 
server responding
with the IP of that website. Another protocol, built on top of HTTP is HTTPS, standing
for hypertext transfer protocol secured. This provides another layer of security using digital
certificates which are given to website owners by certificate authorities. The certificate is given
if and when the website passes certain security measures tested by said authorities. This layer
increases security in that the user will now check the authenticity of the digital certificate of
the website, which the DNS will provide to the user, before exchange of website files.

The user does all this exchanging of website files by means of a web browser. A web browser is a
piece of software into whose address bar the URL of the desired website is typed in. The web 
browser then requests the DNS for the website with that URL, if it exists, the 
DNS 
gives the browser the site's digital certificate of the site. If the certificate is authentic,
the browser requests the webpage's IP address. Otherwise, the user is warned and asked if they want
to proceed. In HTTP, this exchanging of certificates is absent and the webpages are exchanged
with no intermediary security phases.

Websites are written in a programming language called HTML, standing for hypertext markup language.
HTML uses ``tags" to define colour, layout etc.
The primary function of web browsers is to render and display to the user, the website written
in this language. Modern web browsers provide some additional functions, such as they allow the
user to bookmark certain websites and declare certain sites as favourites. The user's browsing
history will also be recorded by the browser to make backtracking easier. Multiple tabs are offered
for simultaneous browsing. Cookies are temporary files that websites store on the user's machine,
managing such files is done by the browser. Navigation tools such as going to the previous webpage
or to the next and such is provided by the web browser, alongside the most basic function of 
allowing the user to type in their desired address.

Cookies are files on the stored by a website on the user's device. They can be of two types,
session cookies and persistent cookies.

Session cookies are created when a user enters a webpage, and are deleted as soon as the user 
exits. An example of usage of such cookies is in online shopping websites, where the ``cart" of
the user is stored as long as the user is in the site itself, once the user exits, all that they
had stored in their shopping cart will be lost, as the session cookie storing that data is lost.

Persistent cookies are not deleted once the user exits the webpage, they are permanent. However,
such cookies have expiration times/dates, beyond which they will be deleted. These are used to
store users' personal data and allow for automatic logins for the user on certain websites. They
may also be used to store the user's preferences and such.

\subsection{Digital currency}

A digital currency is that which exists electronically and not physically. An example of such is
Bitcoin, which is a form of cryptocurrency. Banks are centralised authorities, in that there is
an authority at the centre of it all that oversees all transactions. Cryptocurrencies are 
decentralised, in that there is no central authority. Transactions of cryptocurrencies are kept
track of using a system called blockchain. A blockchain is essentially a list of all records
of transactions. These are called digital ledgers. With each transaction, a record is appended
to the blockchain, including a digital signature with time and date of transaction, once this
data has been entered into the blockchain, it cannot be tampered with. If one wishes to view
the data, they may, but the data that has been entered into the blockchain has been encrypted,
making it very difficult to change the data.

\subsection{Cyber security}

When browsing the internet, users are subject to potential perpetrators, in that they may intercept
and eavesdrop on users' data exchanges. Perpetrators use many strategies to do so. One such 
strategy is a brute force attack.

A brute force attack, in general, means trying every possible combination of something until it
works. This is mostly done for getting the user's password, where every possible combination of
every letter, number and symbol is tried until the correct permutation is stumbled upon.

Perpetrators may intercept users' data by use of packet sniffers, installed onto routers through
which data packets pass. Packet sniffers are software which examine data packets which pass by
to check for any possibly useful data. If contained data seems useful, this packet will be sent
back to the perpetrator.

Using a botnet, perpetrators may attempt a distributed denial of service (DDoS) attack. This is
achieved by sending malware (bad software, literally) to many devices to use those devices as
``bots". Each bot will then target a certain web server and send to it multiple requests, 
simultaneously. A web server can only handle so many requests, as a result the web server crashes.
Such attacks can also be done on other pieces of network hardware other than web servers.

Hacking is the act of attempting unauthorised access to data. The perpetrator who does this is 
called a hacker. They do so by means of brute force attacks, exploiting vulnerabilities in systems
or networks etc.

Malware is any form of malicious software designed to disrupt user's computer or data. Examples
are: virus, worm, trojan horse, spyware, adware and ransomware.

\begin{itemize}
	\item Virus: A computer program installed into unassuming user's hard drive. Such software
		replicates itself, causing the user's device to slow down. It may corrupt the user's
		files and use up all of the user's primary memory, resulting in system crash.
	\item Worm: Another self-replicating software, however such software is meant to find and
		exploit vulnerabilities in a network, replicating itself in those ``holes". This uses
		up the network bandwidth, clogging and slowing down the whole network.
	\item Spyware: Software installed onto user's device for the purpose of spying on the user's
		actions is called spyware. An example is a keylogger, which records all key presses on
		user's keyboard. This may result in the perpetrator gaining access to victim's passwords.
	\item Trojan horse: A trojan horse is meant to be a disguise for other malware. The trojan
		horse looks to be like any other game or such, yet once it has been run, it releases
		malware into your system.
	\item Adware: Software that creates pop ups and banner advertisements on user's screen when
		they are online. These may be annoying, and these are done because the makers of such
		software are paid by whatever company they are advertising.
	\item Randsomware: This software encrypts the user's data and restricts the user's access to 
		it, a sum of money is demanded from the user in exchange for their data. They sometimes
		threaten public display of user's data if money is not duly paid.
\end{itemize}
Pharming is done by perpetrators to gain access to user's personal data. Perpetrators install
malware into user's machine, and whenever they visit certain websites, they are sent to another
website, which, visually looks identical to that which they wanted to go to. Here, they put in
their login details, which the perpetrator now knows and can access.

Phishing is similar to pharming, in that the user is sent to a fake website which mimics a real 
one. However, in phishing the user is sent an email, which contains a fake link, which mimics a
real website. The rest of the process is identical to pharming.

Social engineering consists of people being decieved into providing perpetrators their personal
data.

Such threats can be undone by some strategies. 

Users may utilise access levels, which restrict user's access to certain files on a device 
depending on the user's ``class". An administrator may have access to any and all files on a 
system, a user of that system may not have access to all of it.

Anti malware is a kind of software which looks for patterns of actions used by malware, if any
such applications exist on the user's system, it is reported to the user and action is taken. Anti 
viruses are such software which locate, isolate and confirm whether the software is malicious. Anti
spyware uses the same strategy, but specifically for spyware.

Authentication steps can be taken by the user. The user may set their password consisting of 
seemingly nonsense, using symbols letters, uppercase lowercase and numbers, making it harder
for perpetrators to guess it. An example may be:
\begin{center}
\begin{BVerbatim}
Kjalf89&(&!)YQOEnv,
\end{BVerbatim}
\end{center}
The user may also set up two-step verification, where, during login, the user is asked to do some
thing which only they can do. They may have to enter a code which was sent to a device which only
they have access to.

The user may also set up biometric passwords, where biological data unique to the user only
must be input. These require biometric devices which can scan, for example, the user's fingerprint,
retina etc.

It has been seen that older software tend to have more vulnerabilities which perpetrators can
exploit. Hence, user's should set up systems to automatically update their software to prevent
being victims of such exploitation.

To avoid being phished or pharmed, users must check spelling and tone of emails, emails with
typos and strange grammer and tone tend to be malicious.

Attached to links, the URL can be checked by hovering one's cursor over the link. The URL must
be authentic, for the user to confidently click on it.

Firewalls scan incoming and outgoing data from a user's system, and criteria can be set for the 
firewall to examine said data. This prevents the user from downloading malicious software onto
their machine.

Privacy settings can be set by the user on certain accounts to not have their information be sold
out to companies.

Proxy servers can be used by web servers to act as a barrier to the web server. The proxy server
examines each request before passing it through to the web server. A proxy server prevents 
multiple requests from the same IP, and when the server is bombarded with requests, it passes
the requests through at a slower rate to prevent crashing.

The HTTPS protocol uses either the secure sockets layer (SSL) or transport layer security (TLS)
to determine the security of websites. Checking a website's URL before accessing it can be
useful in that, one can check whether the website is secured. This usually boils down to checking
whether the website URL has a padlock beside the address bar and that the protocol is HTTP or
HTTPS.
