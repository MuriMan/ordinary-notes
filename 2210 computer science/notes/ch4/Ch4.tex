\section{Software}
\subsection{Types of software and interrupts}

\textbf{System software.} System software provides the services that the computer requires, 
this includes the operating system and utility software. Utility software provides ``housekeeping``
services, such as clean up and defragmentation of drives.

\textbf{Application software.} Application software provides the services required by the user,
examples include word processors, spreadsheets etc.\\

An operating system provides an interface for the user, through which the user may enter their 
inputs and outputs. Interfaces may be in the form of graphical user interfaces (GUIs), command line
interfaces (CLIs) and natural language interfaces (NLIs) which allows the user to enter any text
commands or speak them, which the OS then analyses and performs.

It manages the user's files, in terms of file creating, storage within directories and movement
of those files amongst created and deleted directories.

The operating system manages the user's peripherals\footnote{Input and output devices.} and their
corresponding driver software, which translates data from the peripheral to the computer and
vice versa.

A computer uses memory to store data. The movement of data to and from hardware, making sure of
memory being correctly allocated, and preventing two applications from accessing the same memory
location is the job of the OS.

The OS handles which processes are to be processed when by the CPU, so when multiple applications
are running at the same time, the OS is that which controls whose operation is to be executed next.

It is the OS which is the platform on which applications run, it fetches instructions from the
application and executes them.

The OS provides system security by encrypting data and whatnot.

The OS handles the users and accounts of those users of a certain computer. Using the password
for each user, the data of the multiple users are kept secure from one another, unless stated
otherwise.

Understand that, applications run on the operating system, which runs on the firmware, which runs
on the bootloader, which runs directly on the hardware.

Interrupts are generated when something happens which needs the processor's attention. When an
interrupt is generated it is put into a queue according to its priority, where the sequence of
instructions is:

\begin{enumerate}
	\item When the processor finishes its current fetch-decode-execute cycle, it checks the
		interrupt queue.
	\item Sees whether there is an interrupt with a higher priority than the next schedules FDE
		cycle.
	\item If there is, fetch the interrupt.
	\item Check source of interrupt.
	\item Call the relevant interrupt service routine (ISR), which is a sequence of instructions
		that handle the interrupt.
	\item When finished, a higher priority interrupt is called.
	\item If there is no such interrupt, the next FDE cycle is executed.
\end{enumerate}

Interrupts may be caused by software or hardware. Software interrupts consist of problems in code
such as division by zero and two programs trying to access the same memory location. Hardware
interrupts consist of simultaneous peripheral input, pressing the mouse and the keyboard at the
same time is an example.

\subsection{Types of programming language, translators and integrated development environments (IDEs)}

A high level programming language is that which uses English-like keywords. Python, Java and
VB.NET are examples. Such languages are easier to read, write and amend for humans. Instructions
in such languages are easier to debug, and are machine independent, i.e., portable. However,
these languages must be converted to a low-level language before it can be run. One statement
in a high level language can represent multiple low level languages. However, high level languages
cannot directly manipulate hardware.

A low level language is of two types, machine code and assembly language. A computer processes
machine code, which is in binary. Any high level language is converted to machine code before
being executed. Different computers may have different machine code, meaning that such code is
non portable.

Assembly language is an in between stage that use mnemonics to represent code, all high level
languages are converted to assembly before being converted to machine code. Coding in assembly
is hence faster as there is no in between stage before being converted to machine code, and such
code can be used to directly manipulate hardware.

The conversion of assembly language to machine code is done by a piece of translator software 
called an assembler. A compiler is also such a translator software, which translates the entire
code into machine code in one go, taking much time and also reports all errors in code at once.
An interpreter is translator software that translates code to machine code line by line, as they
are executed and stops execution when an error is found. Translating in this method tends to be
faster.
