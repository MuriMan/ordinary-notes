\section{Hardware}
\subsection{Computer architecture}

The central processing unit (CPU) is where all instructions and data are processed, from inputs
to produce outputs. 

Microprocessors are processing units 
on single chips
which are integrated circuits made to perform many of the
functions of a CPU, but to a limited degree.

In a computer with the Von Neumann architecture, the CPU consits of two units, the arithmetic logic
unit (ALU) and the control unit (CU). It has five registers: the program counter (PC), 
memory address
register (MAR), memory data register (MDR), current instruction register (CIR) and accumulator
(ACC). It also has the address, data and control buses.

In the Von Neumann architecture, instructions are processed in a fetch-decode-execute (FDE) cycle.

Data and instructions stored in the the random access memory (RAM), such instructions or data may
be stored in the hard drive, before processing which they will be brought into RAM. The PC is
a register which stores the address of the next instruction to be processed. This address, during
the fetch stage, must be brought into the MAR, which is done by the address bus. The address of
the instruction, now in the MAR, is used to find the data/instruction in RAM. The address bus
goes to that location and sends the data stored there (be it an instruction or some values) to the
MDR using the data bus. Using the data bus, the MDR sends the instruction to the CIR, which is 
part of the control unit, which is responsible for decoding the instruction.

Once the CU has recieved the instruction, it decodes it using an instruction set which is the set
of all commands that the CPU is capable of executing. These commands are usually in machine code.

The execute stage may involve some mathematical calculations, done by the ALU. Any values needed
to be stored temporarily are stored in the ACC.

The CU synchronises all the parts of the CPU to do what they need to, using the control bus.

Each CPU has a cache, multiple cores and an internal clock. 

Each core in a CPU consists of the 
parts described above, hence the more the cores, the more instructions that can be processed
simultaneously, increasing CPU performance.

The internal clock controls the speed at which instructions are processed. A clock speed of 1 Hz
means 1 instruction is processed per second. The more the clock speed, the faster the CPU. However,
increasing the clock speed beyond a certain value can cause overheating of the CPU and result in
damage.

The cache is where instructions most commonly performed by the CPU is stored. The larger the 
capacity of the cache, the more instructions that can be stored, more instructions can be accessed
faster, hence increasing CPU speed.

Embedded systems are used to perform dedicated functions, such as domestic appliances, cars, 
security systems, lighting systems or vending machines. This is different to a personal 
computer in that these can perform many different tasks and are not usually hyper-specific.

\subsection{Input and output devices}

Input devices are those used to enter data into a computer system, including text, images and 
sound.

Barcode scanners are input devices which throw light onto barcodes, and depending on which parts
are reflected and which are not, scan the information stored in the barcode into the computer.

Digital cameras use light and colour sensitive cells to form pixels, which, when arranged as a
matrix can form an image.

Keyboards have buttons, called keys, which, when pressed, signal the computer that a certain letter
has been input.

Microphones have sensors that produce different electric signals depending on the sound level 
around it, which can be input into the computer.

Optical mice work using a red-light source and a sensor to pick up said source. Using the changing
patterns of light sensed, the location of the mouse is calculated and inputted into the computer.

A quick response (QR) code scanner does the same as a barcode, but instead of a linear series of
dark and light lines, a QR code consists of dark and light squares on a matrix.

Touchscreens are of three types: resistive, capacitive and infrared. 

Resistive touchscreens have two layers, when pressure is applied, the two layers come in contact
and an electric circuit is completed, the position of contact can also be calculated. Such screens
do not support multi-touch, give poor visibility in sunlight, and are easily scratched. One must 
also press down quite hard to use such screens. However, they are dust and water resistant, and
can be used with anything (gloves, stylus, whatever).

Capacitive touchscreens have an electric field around them. When a finger is place in said field, 
the field is disturbed and the position of the finger can be calculated. Such screens have good
image clarity, even in sunlight. They are resistance to scratched and support multi-touch. However,
they only work with bare fingers or special styluses and are sensitive to electromagnetic radiation
as those too disturb the electric field.

Infrared touchscreens consist of infrared rays passed across the screen in a matrix pattern. When
a finger touches the screen, these rays are blocked and the sensors which do not recieve their 
respective rays can be used to calculate the position of the fingers. Such screens allow 
multi-touch, are durable and work with scratched or cracked screens. Contrastingly, they are 
water/moisture sensitive, accidental activation is possible if anything blocks the screen and
are sometimes affected by light.

Two dimensional (2D) scanners are used to scan flat documents, done by lighting the document 
producing an image which is made electric by photosensitive cells.

Three dimensional (3D) scanners scan solid objects and produce digital images of actual objects,
they can be used in computer aided design (CAD).

The result of computer processing is shown by output devices.

Actuators are output devices that consist of solenoids whose electromagnetic fields and electric 
fields result in a force
acting on something.

Digital light processing (DLP) projectors pass light through micromirrors whose arrangement
and number define the resolution of the digital image. The light after passing through these 
mirrors passes through a colour filter to produce colour on the projected image.

Inkjet printers have print heads which spray droplets of ink onto the paper to form characters.
Ink cartridges from which they derive the ink, colours and all (using magneta, cyan and yellow 
combinations). A mechanism to move the print head side to side and a paper feed which simply gives
the printer paper to print on. Such printers use either thermal bubble technology, which uses
heat to partially vapourise ink to form a bubble and pass some of it onto the paper. When the
bubble collapses, negative pressure draws more ink into the print head. Piezoelectric technology
uses crystals, which, when they vibrate, force ink onto the paper.

Laser printers use ink powder instead of liquid ink. The paper to be printed upon is given an 
electric charge opposite to that in the ink, in places where the printing must occur. As a result,
ink sticks to the paper. Once printing is completed, the charge on the paper is removed to prevent
paper from sticking to the charged ink drum inside the laser printer.

Light emitting diodes, LEDs, can be arranged with red, blue and green components to create colours
in light. Different amounts of different colours can result in images being produced.

Liquid crystal display (LCD) projectors pass light through red, green and blue mirrors. LCD screns,
depending on the amount of colour on each pixel, block and allow light to pass through and finally
it all converges to form an image through a prism.

LCD screens use changed electric fields to produce different colours. They are backlit by either
LEDs or CCFL technology.

Loudspeakers convert stored sound values in the computer to physical sound waves by using those
sound values to vibrate a physical cone which can then produce corresponding sound waves.

3D printers can print directly and additively, moving the print head wherever it requires and 
adding some material in those places. Sometimes, the material is first printed as a powder and then
it is all made sturdy using glue. They can be used to make custom parts, prosthetic limbs, art, etc.

Sensors are input devices which read conditions of their surroundings and convert them to digital
values using a digital to analogue converter (DAC).

Acoustic sensors are like microphones which take readings of sound levels in surroundings.

Accelerometers measure tehe change in velocity using a piezoelectric cell whose output varies
with change in velocity.

Flow sensors produce output based on the amount of fluid passing around it per unit time.

Gas sensors use various methods to output the amount of the gas being monitored.

Humidity sensors measure the water vapour in a sample of air based on the conductivity of that 
sample.

Using ultrasonic or capacitive methods, level sensors sense the depths of liquids or caverns or
whatever.

Light sensors use photoelectric cells to produce an output depending on the level of light being
sensed.

Magnetic field sensors output the change in magnetic field.

Moisture sensors measure water levels in soil based on resistance of sample.

pH sensors use changes in voltages in sample to output a pH value.

Pressure sensors are transducers which generate different electric currents based on pressure
applied.

Proximity sensors sense the presence of a nearby object.

Temperature sensors produce signals with change in temperature.

\subsection{Data storage}

Primary storage is that which can be directly accessed by the CPU. It consists of the RAM and the
read only memory (ROM). The ROM contains instructions and programs to boot up to computer and the
BIOS. RAM is where instructions to be processed are stored. Hence, RAM is an ever-changing part
of storage, and is volatile in that whatever is stored in RAM will be lost once the computer is
powered down. The ROM is non-volatile and the data stored in it never changes.

Secondary storage is that which is never directly accessed by the CPU, it too, is non-volatile.
Permanent data is stored as secondary storage.

Secondary storage is done in magnetic, optical and solid state storage methods.

Magnetic storage consists of disks called platters on which, magnetism is used to store data. The
disk is separated into sectors and tracks and the disk spins to allow data to be read and written.
An electromagnetic head is used to read or write the data by magnetising dots on these tracks
and sectors, an example of such storage are hard disk drives.

Optical storage utilises lasers to read and write data from a circular disk. Writing data consists
of burning physical pits into the disk itself using the laser. Reading data is also done by a 
laser using the pits and lands and the data that they store. Examples of such storage include
digital versatile disks (DVDs), compact disks (CDs) and Blu-rays.

CDs, read/write DVDs (DVD-RW) and Blu-ray disks have a single track, whereas DVD-RAM have multiple
concentric tracks.

Such storage media require moving parts, which can go wrong. They are not very portable either as
the read-write equipment for the technologies can be quite heavy.

Solid state storage lacks moving parts and uses semiconductor chips to store bits. Such storage
is often called flash memory. Cells and transistors laid out in a grid, parallely in NOR structures
and serially in NAND structures store the data. Using currents sent through the control gates of
transistors, each transistor can hold the value of 0 or 1.

Hard disk drives are cheaper, and longer lasting than SSDs. They are also trusted technology.
However, SSDs need not get upto speed, have faster read/write cycles, run quieter and cooler,
are more portable and lighter, and are more compact.

When there is not enough space in RAM to hold all the data that it must for a certain task, virtual
memory is created in the secondary storages. The excess data to be stored in this storage is 
structured into data structures called ``pages". These pages are transferred back to the RAM
when need be.

Cloud storage consists of servers in a remote location away from the user, on which the user may
store what they want. In comparison to local storage, that which is stored in the cloud can be
accessed from anywhere if one simply has an Internet connection. Data stored locally can only
be accessed from the computer it is stored on and the network(s) it is connected to. For 
businessses, companies providing cloud storage can be more economical than storing their things
themselves, sustaining their servers themselves and hiring personnel to do so, themselves. Cloud
storage may be risky as such servers can be tapped by eavesdroppers.

\subsection{Network hardware}

To access networks\footnote{A network consists of routers, devices and transmission media},
such as the Internet, a device requries a network interface card (NIC). NICs may be wired or 
wireless.

During manufacture, NICs are assigned media access control addresses (MACs), which are unique for
each device and include the manufacturer's code and the device specific serial code. It falls into 
the
following format:
\begin{center}
	\begin{BVerbatim}
	XX:XX:XX:XX:XX:XX
	\end{BVerbatim}
\end{center}
or,
\begin{center}
	\begin{BVerbatim}
	XX-XX-XX-XX-XX-XX
	\end{BVerbatim}
\end{center}
where each \verb{XX{ is an 8-bit hexadecimal number.

An internet protocol (IP) address is used to identify each unique device on the same network. Such
addresses may be static, in that they will never change, or dynamic, in that they change each time
the device connects to the network. IP addresses have two versions, IPv4, which is now quite old
and deprecated, consisting of four 8-bit denary numbers seperated by a period (\verb{.{). IPv4
falls into the following format:
\begin{center}
	\begin{BVerbatim}
	XX.XX.XX.XX
	\end{BVerbatim}
\end{center}
IPv6 falls into the following format:
\begin{center}
	\begin{BVerbatim}
	XXXX:XXXX:XXXX:XXXX:XXXX:XXXX:XXXX:XXXX
	\end{BVerbatim}
\end{center}
where \verb{XXXX{ is a sixteen bit hexadecimal number.
