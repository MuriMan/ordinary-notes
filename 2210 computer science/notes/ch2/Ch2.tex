\section{Data transmission}
\subsection{Types and methods of data transmission}

Large amounts of data, before transmission, is broken into smaller pieces called packets. 

A packet consists of a packet header, a payload, and a packet trailer. The packet header holds
the packet number, required for rearrangement of the packets after complete transmission, the
sender's address and the destination address. The payload is the data being transmitted and the
trailer holds any error-checking systems and also notifies the reciever that the packet has ended.

Data is transmitted across networks, which consist of multiple devices including routers. Routers
decide the route of a packet, which it decides based on which route is busy. As a result, some
packets may have taken a smaller route and may have arrived before others, causing them to be out
of order. The recieving device uses the packet numbers to rearrange the transmitted data once the
last packet has arrived.

Data can be transmitted as serial, parallel, simplex, half-duplex or full-duplex.

Serial transmission is the transmission of data in one bit at at a time. Data as a result, arrives
in order, interference is less likely due to there being only one wire and hence is cheaper as only
one wire need be bought and used. Data is also less likely to be skewed for the same reasons. 
However, transmission in this method is slow, and additional data called start bits and stop bits
may need to be sent to inform the recieving device when the transmission has started and stopped.

Parallel data transmission consists of bits being transmitted simultaneously, across multiple
wires. Data transmission in this method is quicker and since computers transmit data in parallel,
there is no need for conversion. However, since multiple wires are being used the data may arrive
out of order and may be skewed. Interference is more likely as well. Multiple wires are also
pricey.

Simplex data transmission is where data can only be transmitted in one direction.

Half-duplex transmission is where data can be transmitted in both directions, but not 
simultaneously.

Duplex transmission is where data can be transmitted in both directions, simultaneously.

The Universal Serial Bus (USB) interface is an industry standard used to transmit data in serial.
Such interfaces have only one correct connection, meaning no errors in connection can be made. The
speed of transfer is also high in such interfaces. Howewver, a USB cable can only be 5 metres long,
beyond which extenders must be used. Though transmission is fast, it is not as fast as ethernet.

\subsection{Methods of error detection}

Errors can arise during data transmission due to interference, which result in data loss, gain or
change. So, to ensure correct data has been transmitted, we must check for errors.

Parity bits are used at the beginning or end of every byte, leaving 7 bits of actual data. Odd or
even parity may be used. The method consists of counting the number of 1-bits in the 7 bits of 
data, if the parity is odd and the number of 1s is even, the parity bit is set to 1 to make the
number of 1s in the number odd. Same applies for even parity. Below are examples where the leftmost
bit is the parity bit.

\vspace{.2cm}
\begin{minipage}{.5\textwidth}
\begin{flushright}
	\fbox{1} \fbox{1} \fbox{0} \fbox{1} \fbox{1} \fbox{1} \fbox{0} \fbox{1}

	\fbox{0} \fbox{1} \fbox{0} \fbox{1} \fbox{1} \fbox{1} \fbox{0} \fbox{1}
\end{flushright}
\end{minipage}
\begin{minipage}{.4\textwidth}
	even parity

	odd parity
\end{minipage}
\vspace{.2cm}

The checksum method consists of using a certain algorithm to find a value using the data 
transmitted. The value is calculated on the sender's end and transmitted with the data, the 
reciever also calculates it and compares with the value the sender transmitted, if they match, data
is errorless, otherwise an error is there in the data.

Echo checks consist of data being sent to reciever, and the reciever sending it back the to sender.
The sender compares it's recieved data with that which it sent, if an error is found, the data is
re-sent.

Check digits are used to check for errors during data entry. The check digit may already have been
calculated and may exist in a database. During entry, the check digit is recalculated and compared
with that stored in the database, if they are not the same, wrong data has been entered. This is
mostly used during entry of data using barcodes and ISBN, the standard for book codes.

The automatic repeat query (ARQ) can be have negative or positive feedbacks. ARQ using negative
feedback follows:
\begin{enumerate}
	\item Sending device transmits packet
	\item The reciever checks for errors
	\item If no errors are found, no further action is taken
	\item Otherwise negative acknowledgement is sent to sender.
	\item Sender recieves negative acknowledgement and re-sends the data.
	\item A timeout is set by sending device during transmission, if it recieves no acknowledgement
		after timeout, it will stop listening for acknowledgements.
\end{enumerate}

Using positive feedback:
\begin{enumerate}
	\item Sending device transmits packet
	\item Recieving device checks for errors
	\item If none found, positive acknowledgement is sent
	\item If no acknowledgement sent within certain time, data is resent, this may happen for a set
		number of times before data transmission is stopped.
\end{enumerate}

\subsection{Encryption}
Data must be encrypted during transmission so as to protect important data from being tapped.
Encryption is done upon plain text to turn it to cipher text using encryption key. 

Symmetric
encryption uses the same key to encrypt and decrypt, the key is transmitted with the data to
enable the recieving device to decrypt the message. 

Asymmetric encryption uses public and private
keys, which are generated together and a certain user's public key can encrypt the message in
such a way that the decryption of that message can only be done by that user's private key. So,
before transmission, the reciever sends the sender his public key, the sender encrypts data using 
it, and transmits the data. The reciever decrypts it using his private key.
