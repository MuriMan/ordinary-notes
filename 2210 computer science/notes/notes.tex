\documentclass{article}

\usepackage{amsmath, tcolorbox}
\setcounter{tocdepth}{2}
\title {Cambridge GCSE Notes \\ 2210 Computer Science }
\author {Abrar Faiyaz Rahim, degrees pending}
\date {}

\begin{document}
\maketitle \newpage
\section*{Introduction}
What follows is my compilation of notes I used to get through my Ordinary Level GCSE 
exams, in the May-June session of 2025. I release all these notes to the public so as to 
help combat the tragic ``coaching culture" surrounding GCSE exams. This set of notes is 
written to be used in accordance to the coursebooks published by Cambridge, and is of
little use by itself.

My despiction of coaching culture arises from the fact that it ruins student life and
to a large extent the academic potential of students. A typical student has school in the 
day and in the evening they are made to sprint to and from coaching centres, often 
multiple teachers for the same subject all as a result of peer pressure and they come back home at 
9 or 10 at night. If they have been given homework, they must sacrifice their sleep to
complete these assignments. Students cannot, as a result, study by themselves -- ruining
their potential.

There exist accessible resources which are more than enough for a candidate to ace their 
exams, without any aid from the financially-minded coaching sharks. One of the services
provided by these teachers are ``compiled notes", exchanged for money, 
further inflating the price of education. So, in retaliation I release these 
notes as open-source and free to distribute.

Yet coachings are not entirely evil, students who struggle in certain subjects may ask 
for the aid of teachers of those subjects but it is meaningless to go to a different 
teacher for each and every subject.

These notes are condensed, written in sequence of the Cambridge specifications.

\section*{Suggestions for readers}
These notes are not at all stand-alone resources that will magically help you to get 
through your exams. I suggest purchasing and utilising the Cambridge coursebooks, and
read the chapters from there before referring to these notes, especially if the topic
in question is absolutely novel to you. 

YouTube has excellent resources, lectures galore
at your discretion, simply type in ``GCSE" alongside whatever topic you need to watch
the lectures on.

Lastly, for practice of questions, past papers both topically and yearly compiled are
available for purchase at bookstores and for free online (topicals can be found on
physicsandmathstutor.com).
\tableofcontents \newpage
\section {Data Representation}
\subsection{Number systems}

\subsubsection{Computers and binary}
The number system consisting of digits 0 through 9 is called the denary or decimal system
(10 digits). Computers use another number system called binary, consisting of digits
0 through 1. This is done such that computers are able to pass the data through logic 
gates and can be stored in registers.

\subsubsection{The binary, denary and hexadecimal systems}
The number of digits in a number system is the base of that system; denary is base 10;
binary is base 2; hexadecimal is base 16.

\subsubsection*{Conversions}
From positive denary to positive binary:
\begin{enumerate}
	\item Perform short division on given denary number, taking note of the remainders.
	\item Write the remainders from bottom to top, resulting in the binary number.
\end{enumerate}
From positive binary to positive denary:
\begin{enumerate}
	\item Write the binary number with their \textit{place powers}.
	\item Sum the products of each binary digit with the place power, resulting in the
		converted denary number.
	\item Write the remainders from bottom to top, resulting in the binary number.
\end{enumerate}
From positive denary to positive hexadecimal:
\begin{enumerate}
	\item Convert given number to binary
	\item Split resulting binary number to four-bit parts.
	\item Convert the four-bit binaries to denary.
	\item 1 = 1; 2 = 2; ... 10 = A; 11 = B; 12 = C; 13 = D; 14 = E; 15 = F.
	\item Arrange resulting digits side-by-side.
\end{enumerate}
From positive hexadecimal to positive denary:
\begin{enumerate}
	\item Convert each given hex number to denary using the index in step 4 of 
		denary-hex.
	\item Convert each denary number to binary.
	\item Arrange the resulting four-bit binary pieces, producing binary result.
\end{enumerate}
From positive hexadecimal to positive binary:
\begin{enumerate}
	\item Convert to denary.
	\item Convert to binary.
\end{enumerate}
From positive binary to positive hexadecimal:
\begin{enumerate}
	\item Convert to binary.
	\item Convert to denary.
\end{enumerate}

\subsubsection{Uses of the hexadecimal system}
The hexadecimal number system is used to make life easier for humans dealing with 
bare low-level computer code. Hexadecimal requires less digits, and are easier to compare
with the naked eye. Data errors can be easier to find when looking at this shortened form
of binary, hexadecimal.

\subsubsection{Binary addition}
To add two binary numbers, refer to the following:
\begin{itemize}
	\item $0 + 0 = 0$.
	\item $1 + 0 = 1$.
	\item $1 + 1 = 10$. (the one is carried on)
	\item $10 + 1 = 11$. (the one is carried on)
\end{itemize}

Sometimes, the addition results in an extra bit, which \textbf{overflows} off. This is
because computers have predifined limits to which it can store its numbers (16, 32 bits)
and when a value outside this limit is returned, it is not stored and an overflow error
occurs.

\subsubsection{Logical binary shifts}
When performing logical shifts, we simply move the bits of a binary number to the right
or left depending on what is required. We ``delete" and hence lose the leftmost (most 
significant) or rightmost (least significant) bit depending on the direction of the shift 
performed.

Shifting right means dividing by two.

Shifting left means multiplying by two.

\subsubsection{Two's complement}
Two's complement is a method used to represent negative binary numbers. We simply
convert given denary number to binary (if need be), invert all the bits and add $(1)_2$ 
to the result.

\subsection{Text, sound and images}
\subsubsection{Text}
Text is converted to binary so that a computer can process it. It does so by converting
each character into an integer, as defined in the \textbf{ASCII} standard (American 
Standard Code For Information Interchange) and subsequently into a binary number.

\textbf{Unicode} is another such standard, which allows a greater range of characters, in
various languages, as a result it also requires more bits per character.

\subsubsection{Sound}
Sounds are composed of waves. When we record values of the sound, we do so at set 
intervals, this process is called \textbf{sampling}. The more samples taken per unit
time, the more accurate the sound recorded will be, i.e., higher the \textbf{sample rate}
the greater the sound quality.

The sound values, which are usually denary numbers, can be converted to binary and stored
into a computer.

The \textbf{sample resolution} is the number of bits allocated per sample value. So, the 
larger the sample resolution, the more the amount of digits that can be stored into a 
file. Thus, the higher the sample resolution, the higher the sound quality.

The file size of a sound file increases, with increased sample rate and sample 
resolution, that means storing high quality sound requires more space that low quality
sound.

\subsubsection{Images}
An image is composed of \textbf{pixels}. The computer stores these pixels by processing
them to binary, by assigning a binary number to a certain colour. 

The \textbf{resolution} of an image is the number of pixels stored in it. Usually in the
format: width $\times$ height.

The \textbf{colour depth} of an image is the number of bits allocated for each pixel
of the image. Higher the colour depth, the more the number of colours that can be
displayed.

Higher quality images result in larger file sizes as resolution and colour depth are 
large.

\subsection{Data storage and compression}
\subsubsection{Measurement of data storage}
\begin{itemize}
	\item Bit: 1 or 0. Smallest possible data measurement.
	\item Nibble: 4 bits.
	\item Byte: 8 bits.
	\item Kibibyte (KiB): 1024 Bytes.
	\item Mebibyte (MiB): 1024 Kibibytes.
	\item Gibibyte (GiB): 1024 Mebibytes.
	\item Tebibyte (TiB): 1024 Gibibytes.
	\item Pebibyte (PiB): 1024 Tebibytes.
	\item Exbibyte: (EiB) 1024 Pebibytes.
\end{itemize}

\subsubsection{File size calculation}
To calculate the file size of an image, find the product of the image's width, height, 
colour depth

For the sound's file size, multiply the sample rate, sample resolution and soundtrack
length.

\subsubsection{Compression}
Files can be compressed to:
\begin{itemize}
	\item Reduce storage space needed.
	\item Less transmission times among devices.
	\item Quicker upload and download times.
	\item Requires less bandwidth for file transmission.
\end{itemize}
Compression is of two types:
\begin{enumerate}
	\item Lossy: Reduces file size by permanently reducing colour depth, resolution, or 
		sample rate.
	\item Lossless: Reduces file size without permanent loss of data, using Run Length
		Encoding (RLE). RLE groups similar data together and hence some file space can
		be saved.
\end{enumerate}

\section{Data transmission}
\subsection{Types and methods of data transmission}
\subsubsection{Packets and transmission}
When data is transmitted from one device to another, it is organised into packets. A 
packet consists of three parts:
\begin{itemize}
	\item Header: Consisting of three parts again: destination address, packet number and
		originators address. 

		\begin{itemize}
		\item The destination address is an Internet Protocol (IP) address which is a 
			unique identifier address for every computer connected to the internet.
		
		\item The packet number assigned to a packet helps the recieving computer reorder and
		organise the data sent, because often data may be sent out of order.

		\item The originators address too is an IP address. It is that of the device from which
		data has been sent. It helps to trace origin of transmitted data.
		\end{itemize}
	\item Payload: Consists of the actual data being sent.
	\item Trailer (Footer): Consists of data for any error detection system, and the 
		data to signal to the reciever that this is the end of the packet.
\end{itemize}

Data is transmitted across a network (the internet for the majority of cases). Networks
consist of routers. From one device to another, there exist multiple connected routers
amongst whom there exists multiple paths whicch data packets can take. The routes are
decided by the routers themselves. Packets may arrive out of order as a result but the
packer number stored in the header helps the reciever reorder the recieved data. This is
the process of packet switching.

\subsubsection{Methods of dataa transmission}
There are two methods of data transmission regarding the volume of data transferred:
\begin{itemize}
	\item Serial: Data is transmitted along a single wire a bit at a time. Sequencec is
		maintained, very little interference is likely and cheaper for manufacturer and
		consumer because only requires one wire. 

		However, data transmission is very slow 
		and start and stop bits must be sent additionally.
	\item Parallel: Data is transmitted along multiple wires, multiple bits at a time. 
		Data transmission is quicker, because of the sending of multiple bits at one time.
		No need for conversion for transmission across networks as computers use parallel
		transmissions internally. 

		However, the bits may arrive out of order and skewing is a risk. Interference is 
		likely and errors may arise. Expensive due to more wires required.
\end{itemize}
There are three methods of data transmission regarding the direction of data transferred:
\begin{itemize}
	\item Simplex: Data is transferred in one direction only.
	\item Half-duplex: Data is transferred in both directions, not at the same time.
	\item (Full-)Duplex: Data is transferred in both directions at the same time.
\end{itemize}

\subsubsection{The universal serial bus (USB) interface}
The USB interfaec inclcudes the port, cable, connection and device. Data transmission, 
here is a special type of serial which allows high speed transmissions.

The USB interface is simple and connections can only be made in one way, less errors in
onnecting devices are likely. The speed of data transfer is quite high. It is the 
industry standard so almost all devices are equipped with a USB port. When USB devices are
connected, required drivers for the devices are automatically detected and downloaded.
It does not need its own power source and can be used to charge devices.

The length of a USB cable is limited to a maximum of five metres. Transmission is quick
yet not as quick as ethernet.

\subsection{Methods of error detection}
\subsubsection{Necessity of error checking}
Errors may occur as a result of interference during transmission, consisting of loss, gain
and change of data being transmitted.

\subsubsection{Processes to detect errors}
These errors can be resolved in the following ways:
\begin{itemize}
	\item Parity check: Given data is said to have even or odd parity. Bytes of data are
		sent with a parity bit, which is determined by the data itself. If the number of
		ones in the binary data is even and the parity is even, the parity bit will be 
		one, otherwise zero. Same stands for odd parity.
	\item Checksum: A calculated value is transmitted with transmitted using a certain
		method. The reciever then uses the same method to calculate the value itself.
		If the transmitted and calculated values match, the data is error-free, otherwise
		it is corrupt.
	\item Echo check: Recieved data is sent back to the sender, who compares it with 
		original data. If datas match, no error. Otherwise data is sent again.
\end{itemize}

\subsubsection{Check digits}
Check digits are identical to checksums but the the result of the generating algorithm
is in a single digit. ISBN (International Standard Book Numbers) use check digits and
so do barcodes.

\subsubsection{Automated Repeat Queries (ARQ)}
ARQs work in the following sequence:
\begin{enumerate}
	\item Data is sent to reciever.
	\item Reciever checks errors.
	\item If data free, send positive acknowledgement to sender.
	\item Otherwise send negative acknowledgement and sender re-sends data.
	\item If no acknowledgement is sent beyond the timeout limit, sender sends data again.
	\item Without acknowledgement, data is sent for a set number of times.
\end{enumerate}

\subsection{Encryption}
\subsubsection{The need for data transmission}
Data needs to be encrypted so as not to lose sensitive data to potential hackers.

The original data is called the plaintext, having used an encryption algorithm, usually
with an encryption key, a ciphertext is formed. This ciphertext is then sent across the
network and some method of data decryption is used by reciever.

\subsubsection{Methods of data encryption}
Data can be encrypted in two ways:
\begin{enumerate}
	\item Symmetric: Data is encrypted and decrypted using the same encryption key which is 
		send along with the data itself. Vulnerable method as encryption key can also be 
		compromised.
	\item Asymmetric: Data is encrypted with the senders public key, decrypted with 
		public key. This is the safer method as compromise is unlikely with only public
		key.
\end{enumerate}

\section{Hardware}
\subsection{Computer architecture}
\subsubsection{The central processing unit (CPU)}
It is responsible for the process of data inputted into the computer to turn it into an
output. A microprocessor is present in embedded systems and is an integrated circuit 
which is able to perform many of the functions of a CPU.

\subsubsection{The Von Neumann architecture}
The Von Neumann architecture consists of three stages in a cycle: fetch, decode and 
execute.

\begin{itemize}
	\item Fetch: 
		\begin{enumerate}
			\item Inputted data, instructions and data from hard drive are initially 
				stored and put into the RAM (Random Access Memory).
			\item A register called the PC (Program Counter) stores the address of the next
				instruction to be processed. The address is the next location of RAM.
			\item When an instruction is to be processed, the address from the PC is 
				brought into the MAR (Memory Address Register). The address bus is used for
				the movement of registers.
			\item Using the address stored in the MAR, the address bus retrieves the data
				at the address in the RAM, bringing it back into the MDR (Memory Data
				Register) using the data bus.
			\item Once the MDR recieves the data, which is the next instruction to be
				processed, this data is sent to the CIR (Current Instruction Register). The
				transfer is done by the data bus.
			\item This is the end of the fetch stage, the CIR is part of the CU (Control
				Unit) which is responsible for the second stage: decode.
		\end{enumerate}
	
	\item Decode:
		\begin{enumerate}
			\item Using an instruction set, the CU decodes the instruction stored in the
				CIR.
			\item This is the end of the decode stage, now the execute stage can begin.
		\end{enumerate}

	\item Execute:
		\begin{enumerate}
			\item Here actions required for carrying-out of instructions are done.
			\item Calculations are done by the ALU (Arithmetic Logic Unit). The ALU
				has a register called the ACC (ACCumulator) where any temporary values
				needing to be stored are stored.
		\end{enumerate}
\end{itemize}
The stages in the fetch-code-decode cycle are coordinated by signals transmitted through
the control bus.

\subsubsection{CPU Performance}
CPU Performance is controlled by three main factors: number of cores, clock speed and 
cache.

\begin{itemize}
	\item Cores: The more the number of cores the better the performance as more fetch-
		decode-execute cycles can run simultaneously.
	\item Clock speed: A CPU contains an internal clock which controls speed of 
		processing of instructions. Using overclocking, CPU can process more instructions
		quicker but it can cause overheating.
	\item Cache size: Cache is a type of storage inside and hence near the CPU. 
		The more instructions stored in cache the better as it takes less time than 
		fetching instructions all the way from RAM.
\end{itemize}

\subsubsection{Embedded systems}
Embedded systems are essentially small computers that are built to do a very specific 
task. Examples include the systems embedded into domestic appliances (microwaves, 
fridges, etcetera), vending machines, security systems or lighting systems. They use
microcontrollers in place of a CPU and usually lack some parts of the Von Neumann 
architecture.

\subsection{Input and output devices}

\subsubsection{Input devices}
An input device is that which allows any entering of data into a computer system, 
including text, image and sound. 

Required input devices:
\begin{itemize}
	\item Barcode scanner: Used to scan data encoded into a barcode (a linear image 
		consisting if dark and light lines).
	\item QR (Quick Response) code scanner: Used to scan data incoded into a QR code (an
		image consisting of dark and light squares arranged in a matrix pattern).
	\item Digital camera: Used to take digital images of surroundings.
	\item Keyboard: Used to type in text into a computer system.
	\item Microphone: Used to input sound data into a computer system.
	\item Optical mouse: A pointing device which uses a CMOS (Complementary Metal Oxide
		Semiconductor) to detect movement and maps that movement into the pointer on
		the screen.
	\item Touchscreen: Can be of three types: capacitive, resistive and infrared:
		\begin{itemize}
			\item Capacitive: Voltages are produced at all four corners of the screen,
				any contact by finger or stylus results in change in electric field 
				produced, position of contact can then be calculated.
			\item Resistive: Consists of two layers, when pressure is applied, the layers
				come into contact, completing a circuit and position of contact is
				calculated.
			\item Infrared: Infrared light beams are shot across thes screen in an X-Y
				pattern. When a finger or stylus contacts the screen, the light rays are
				blocked and the position of contact can easily be calculated.
		\end{itemize}

	\item 2D and 3D scanners: 2D scanners are used to scan what is printed onto a piece
		of paper, a document. This is done to digitise the document.

		3D Scanners scan and produce a 3D image of a given object. This can be used in
		CAD (Computer Aided Design) circumstances.
\end{itemize}

\subsubsection{Output devices}
An output device is that which allows for the result of data processing to be understood
by humans.

Examples are:
\begin{itemize}
	\item Actuator: An actuator is a mechanical or electromechanical device such as a 
		relay, solenoid or motor. They are needed to start/stop and open/close mechanical
		parts.
	\item Digital light projector (DLP): Uses millions of micro-mirrors on a small digital
		micromirror device (DMD). 

		Gives high contrast ratios, lasts longer, is quieter,
		gives no issues lining up images, smaller and lighter than LCD projectors, do better
		in dusty or smoky atmospheres.

		Shadows arise with moving images, lack grey components, colour saturation 
		(intensity) not as good as LCD projectors.
	\item Liquid crystal display (LCD) projector: 
		\begin{enumerate}
			\item White light is split into red, green and blue by dichromatic mirrorrs.
			\item These three groups are reflected by chromatic-coated mirrors into another
				set of mirrors;
			\item Lastly they are reflected into a special prism which outputs the image.
		\end{enumerate}

		Give sharper images than DLP projectors with better saturation, less heat 
		is generated and are more efficient.

		Contrast ratios are worse, do not last long, degrade over time (organic nature).
	\item Inkjet printer:
		\begin{enumerate}
			\item Document sent to printer driver which ensures document format.
			\item Check to see if printer is busy.
			\item Data is sent to the printer buffer (temporary memory).
			\item The print head moves laterally across the paper printing text.
			\item Advanced vertically forward and repeated until page is printed.
		\end{enumerate}

		\textbf{Piezoelectric} inkjet printers used charged crystals to eject ink onto 
		paper.
		
		\textbf{Thermal bubble} inkjet printers use tiny resistors to produce heat, form 
		ink bubbles which is then ejected onto paper.

		Small ink cartridges and paper trays make it such that few colour images are
		feasible with inkjet printers.

	\item Laser printer:
		\begin{enumerate}
			\item Same as inkjet.
			\item Same as inkjet.
			\item Same as inkjet.
			\item Positively charged ink sticks to negatively charged printing drum.
			\item Ink droplets stick to negatively charged paper.
		\end{enumerate}

		Can print lots of documents in high quality very quickly, can hold a large amount
		of paper and large amounts of ink.

	\item Light emitting diode (LED) screen: Composed of tiny LEDs, which are either red,
		green or blue. Varying current sent to each LED produces variation in base colour
		brightnesses, producing colours.

	\item Liquid crystal display (LCD) screen: Composed of tiny liquid crystals, making
		up a matrix of pixels affected by changes in electric fields. LCDs themselves do
		not produce any light and so they are backlit using LEDs.

	\item Speaker: Digital (stored) data of sound is passed through a DAC, resulting data
		is passed through an amplifier and subsequently a varying current is made to pass
		through a solenoid which subsequently makes a plastic/paper cone move producing
		sound waves in the air.

	\item 3D printer: Builds up layers, horizontally and then vertically to produce a
		solid object. Uses powdered resin, metal or ceramic. Can be used for prosthesis,
		reconstructive surgery, aerospace, art, parts no longer in production.
\end{itemize}

\subsubsection{Sensors}
These are input devices which read things from their surroundings and convert them to 
digital data using an ADC.

Required sensors:
\begin{itemize}
	\item Acoustic: Microphones that convert sounds into electric signals/pulses.
	\item Accelerometer: Uses a piezoelectric cell whose electric output changes with
		change in velocity of that being observed.
	\item Flow: Measures rate of flow of a moving liquid or gas based on the amount of
		substance passing over the sensor.
	\item Gas: Uses various methods, produces varying output depending on gas present.
	\item Humidity: Measures the amount of water vapour in a sample of air.
	\item Infra-red: Uses a detector which detects infra-red beam, if broken, electrical
		signal is changed.
	\item Level: Ultrasonics are used to check how high a liquid is in a container.
	\item Light: Photoelectric cells are used which produce an output depending on the
		presence and intensity of light.
	\item Magnetic field: Output changes according to change in magnetic field sensed.
	\item Moisture: Measures water level in samples using electrical resistance.
	\item pH: Measures acidity using changes in voltages.
	\item Pressure: A transducer that generates a different current based on pressure
		applied.
	\item Proximity: Detect presence of any nearby object.
	\item Temperature: Uses signals that change as temperature changes to output signals
		that change as temperature changes.
\end{itemize}

\subsection{Data storage}
The memory and storage devices of a computer are in two parts, primary and secondary.

\subsubsection{Primary storage}
Primary storage devices are those that can be accessed directly by the CPU, these include
Random Access Memory (RAM) and Read Only Memory (ROM).

RAM is used to temporarily store some data, such as file data in use and yet to be saved.
RAM is volatile meaning all contents of RAM is lost as soon as the computer is turned
off. RAM can be read from and written to.

ROM consists of the BIOS, cannot be written to and are very little in amount. Data is 
non-volatile, i.e. permanent.

\subsubsection{Secondary storage}
Secondary storage is not accessed directly by CPU and is used for more permanent storage.

\subsubsection{Magnetic, optical and solid-state storage}
Magnetic memory is stored on a magnetic disk, using tracks and sectors on the disk. Data
is read and written using electromagnets.

Optical storage (DVD, CD) uses a laser to dig physical pits and lands onto disks, which 
can be interpreted as data.

Solid-state or flash memory uses NAND and NOR technology to store electric pulses as 
memory, transistors are used as control and floating gates.

\subsubsection{Virtual memory}
When a computer is about to run out of RAM space, it allocates some memory to secondary
storage so as to avoid a system crash, where the memory is located is called virtual
memory.

\subsubsection{Cloud storage}
Data is stored on remote servers, so that users can access this data any time they want.

Allows for backups to be made, almost unlimited capacity, and there is no need to be
carrying a whole storage device on the user's person all the time.

Data stored, however can be hacked. Data stored can also be lost by the cloud company
itself. Large amounts of storage is expensive. Internet speed can affect user experience.

\subsection{Network hardware}
\subsubsection{Network interface card (NIC)}
To be part of a network, a device requires a NIC. 

\subsubsection{Media access control (MAC)}
The NIC holds a Media Access Control
(MAC) address which is unique to that NIC and is assigned by the manufacturer at the time
of manufacture. It consists of the manufacturers identification code followed by the
devices unique identification code in the following format:
\[\textrm{NN-NN-NN-DD-DD-DD}\]
where N is the manufacturer code and D is the device code. The MAC address is usually
in hexadecimal.

\subsubsection{Internet protocol (IP)}
IP addresses are unique addresses assigned by the network onto devices that are part of
the network, each IP of a device to a network is unique inside of that network. 

Dynamic IPs are those where the IP changes each time the device connects to the network.
Static IPs are those where the IP stays the same, even after disconnection and 
reconnection to the same network.

IPv4 was the initial version of the protocol, consisting of four eight-bit numbers 
seperated by periods. An example may be:
\[256.123.145.27\]
IPv6 is the newer version using 128-bit numbers of eight groups seperated by colons. 
Because of the size of each group of an IPv6 address, it is represented in hexadecimal:
\[\textrm{A8FB:7A88:FFF0:0FFF:3D21:2085:66FB:F0FA}\]

\subsubsection{Routers}
Routers are involved in packet switching, and they identify different devices using the
differences in IP addresses.

\section{Software}
\subsection{Types of software and interrupts}
\subsubsection{System and application software}
In general, there are two types of software: system and application.
\begin{itemize}
	\item System: Consists of the Operating System (OS) and utility software (compilers,
		drivers, anti-viruses etc), required for device usage.
	\item Application: Consists of services required by the user: word processors, video
		editors etc.
\end{itemize}

\subsubsection{The role and functions of the OS}
The OS's job is to manage the user's files, handle system interrupts and it provides an
interface between the software and hardware of a device. It manages drivers required for
peripheral input devices, it's the OS's job to manage the multiple tasks being done by
the user and the priority of those tasks. It is the OS's job to manage memory allocation.
It is upon the OS that application software runs, and the OS provides security consisting
of anti-viruses. It also manages the user's accounts.

\subsubsection{Hierarchy dependencies}
The bootloader or the firmware runs diretly on the hardware. It is upon the firmware that
the OS runs on which run applications.

\subsubsection{Interrupts}
Interrupts are basically signals to the processor. Every interrupt has a priority level
and interrupts are handled by the creatively named interrupt handler (IH). The IH 
procedure follows:
\begin{enumerate}
	\item After an FDE cycle is done, the system checks for any generated interrupts.
	\item It compares the priority of the interrupt to the next task to be processed, if
		it is the next step is processed otherwise step 7 is executed.
	\item Stores the less process with inferior priority.
	\item Checks what sent the interrupt.
	\item Calls the relevant interrupt service routine (ISR), which is a procedure that
		handles the interrupt.
	\item When the ISR is done it goes back to the FDE cycle that was stored away.
	\item Execute the the FDE cycle. 
\end{enumerate}
Interrupts are generally of two types: software and hardware. Examples follow:
\subsubsection* {Software:}
	\begin{itemize}
		\item Division by 0.
		\item Two programs accessing the same location at the same time.
		\item Input request.
		\item Output request.
	\end{itemize}
\subsubsection* {Hardware:}
	\begin{itemize}
		\item Data input (key pressed).
		\item Mouse moved.
	\end{itemize}

\subsection{Types of programming language, translators and integrated development
environments (IDEs)}
\subsubsection{Types of programming language}
Programming languages are generally seperated into two categories: high-level and 
low-level.

\subsubsection*{High-level languages}
These are languages which have statements consisting of English words. Examples of such
languages are Python, VB.Net and Java. An example of a statement in a high-level language
is:
\begin{verbatim}
int age = 16;
String username = "Abrar";
\end{verbatim}
A program written in a high-level language is "portable" in the sense that the program
code can be executed on any computer regardless of what device the code was written on.

\subsubsection*{Low-level languages}
Low-level languages are generally of two types once again: machine and assembly language.

Machine language consists of 1s and 0s that only the computer can read, and high-level
languages are translated into machine code before it can be executed. Machine code is
non-portable because code that executes on one device may not on another because of the
differences in manufacturer and OS formats.

\subsubsection{Assembly language}
Assembly language uses mnemonics such as LDD (load) ADD (add) etc. It's still human
readable but less so than high-level languages. It allows very close communication to
hardware, but lots of statements need to be written for a very simple instruction.
Example:
\begin{verbatim}
LDD 8x7917f
ADD 1
STO 8x7919f
\end{verbatim}
\subsubsection{Translators}
To execute program code the code must be translated down into machine code, which is done
by means of usage of translators.

\subsubsection*{Compilers}
The compiler goes through the whole program code, reporting any and all errors before
producing an executable file. If errors are found they are shown to the programmer and
no executable file is produced. 

\subsubsection*{Interpreters}
These are translators that translate and execute the program code line-by-line. If errors
are found the translation process is stopped and the error is reported, as a result not
all errors are reported at once. 

\subsubsection{Compilers and interpreters in context}
Compilers are useful when one is done with building
a program, as an executable file will then be produced. Interpreters are not suitable
because interpreter software will be required to run the code alongside the actual code
making it difficult to distribute.

Interpreters are useful while building a program, as 
errors pop up as the program is being executed. Compilers are unsuitable during the 
building process because all errors must be fixed before execution.

\subsubsection{Integrated development environments (IDEs)}
IDEs are used to write program code. They are complex software that provide all the
utilities needed by a programmer. Including:
\begin{itemize}
	\item Code editors: Used to edit the code itself.
	\item Run-time environment: Used to see the output of the code.
	\item Translators: Discussed in the previous sub-subsection.
	\item Error diagnostics: Tells you where and why an error is occurring.
	\item Auto-completion: Suggests the ending of a command which the user is typing.
	\item Auto-correction: Changes what the user wrote to what the user meant.
	\item Prettyprint: Colours different keywords differently for ease of the programmer.
\end{itemize}

\section{The internet and its uses}
\subsection{The internet and the world wide web}
\subsubsection{What's the difference?}
The internet consists of the infrastructure, i.e. the cables and routers and all the 
devices that are used to connect the devices connected to each other. The network
itself is the internet, in this case its called a Wide Area Network (WAN) and the wide
area in question is the whole world.

The world wide web, is the collective name given to the websites and webpages available
for access through the internet.

\subsubsection{Uniform resource locator (URL)}
A URL is the unique text-based address for a website on the internet. It's typed into
the address bar of a web browser. It consists of three parts: protocol (HTTP or HTTPS),
the domain name (website name, youtube, netflix, etc) and the webpage name (the name of
the page inside the website itself) in the following format:

\[\textrm{protocol://www.domain-name/web-page-name}\]

\subsubsection{HTTP and HTTPS}
HTTP stands for Hyper Text Transfer Protocol and it is by this protocol that webpages
are transferred across the internet to web browsers. HTTPS is simply Hyper Text Transfer
Protocol Secure and it is used to securely transfer these webpages, by use of digital
security certificates.

\subsubsection{The web browser}
The primary function of a web browser is to retrieve webpages (the process is shown
in the next sub-subsection), and to render and display those webpages to the user. All
webpages are written in the Hypertext Markup Language (HTML) alongside Cascading Style
Sheets (CSS) and some active scripts written in languages such as JavaScript. Modern
webpages provide functions such as:
\begin{itemize}
	\item Bookmarking and storing favourite websites.
	\item Recording user's browsing history.
	\item Allowing use of multiple tabs to browse multiple places simultaneously.
	\item Storage of cookies.
	\item Provision of navigation tools.
	\item Providing an address bar.
\end{itemize}

\subsubsection{The process of location, retrieval and display of a webpage}
\begin{enumerate}
	\item The user types in the desired website's URL into the browser's address bar.
	\item The browser sends the URL to the domain name server (DNS), which searches
		for that URL in its database. 
	\item If the URL is found, the DNS sends the IP of that website to the web browser.
	\item Using that IP the browser requests the web server of that website to send the
		specified web page to the web browser (done by HTTP/HTTPS).
	\item Once the webpage is recieved, the browser renders the webpage and displays it
		to the user.
	\item If the URL isn't in the DNS database, the DNS sends the URL to another DNS
		which also checks, if the URL isn't present in any of the DNS databases, the 
		URL doesn't exist and the user is notified.
\end{enumerate}

If the HTTPS protocol is being used, the web browser first asks for a digital 
certificate from the web server and sees if it is authentic, if not the user is informed
that the webpage is not secure.

\subsubsection{Cookies}
Cookies are files that store data on the user's device. They can store data that is
regularly used and hence save the user the boredom of inputting such things repeatedly
(logins, usernames, etc). Cookies are used for the following:
\begin{itemize}
	\item Saving personal details.
	\item Tracking user preferences.
	\item Holding items in an online shopping cart.
	\item Storing login details.
\end{itemize}
Cookies are of two types: session and permanent.
\begin{itemize}
	\item Session: These are temporary cookies that only stay on your device as long as
		you are on a certain webpage, such as an online shopping cart which is on the
		device as long as you are shopping, but as soon as you close the webpage the
		data regarding your shopping cart is deleted from your device.
	\item Persistent: These cookies are permanent and remain on your device even after
		you close the webpage. Examples include the user's preferences on a certain 
		webpage, etc.
\end{itemize}

\subsection{Digital currency}
\subsubsection{Electronic existence}
Currency that only exists electronically and is only exchanged amongst computers is 
called digital currency. Credit cards and mobile banking are examples. 

\subsubsection{Blockchains}
Cryptocurrencies, such as Bitcoin and Dogecoin are decentralised currencies, i.e. no
centralised authority is in charge of keeping track of the transactions made by these
currencies. But the transactions must be kept track of nonetheless, this is done using
blockchains.

A blockchain is a digital ledger, which is encrypted. That means whenever a transaction
is made a new "entry" is added onto the blockchain, hence the transactions are recorded.
Blockchains are irreversible meaning once an entry has been made no one can change it.

\subsection{Cyber Security}
\subsubsection{Threats}
When browsing the internet, there is a chance that data being transmitted across the
network can be intercepted and stolen by potential hackers. They can do so in the 
following ways:
\begin{itemize}
	\item Brute-force attack: Hackers will try to use possible combination of letters
		to maybe stumble upon your username and subsequently your password for a 
		certain account of yours.
	\item Data interception: This is done by use of software known as packet sniffers,
		installed onto routers which can determine whether a packet passing through
		the router is useful, if it is then a copy of the packets are sent to the 
		hacker.
	\item Distributed denial of service (DDoS): Malware (malicious software) is sent to
		many user's devices, all the devices being part of a network. This malware then
		sends requests to a certan web server, and since a server can only handle a 
		certain number of requests, it literally gets overwhelmed and slows down to a
		snail's pace where it cannot handle any requests quick enough.
	\item Hacking: Hackers may exploit vulnerabilities in a network, to gain access to
		data being transmitted across the network; they brute-force attack a user's 
		account, already discussed.
	\item Malware: Can be of various types:
		\begin{itemize}
			\item Virus: Malware that replicates itself and corrupts, and slows down
				victim's computer by using up all available memory.
			\item Worm: Similar to virus, only it clogs up the bandwidth of the network
				it is connected to.
			\item Spyware: Malware that runs in the background as the unaware user
				uses his/her device and the spyware spies on the user, often stealing
				usernames and passwords and hence access to accounts and sends it on
				to the perpetrator. This can be done by means of a keylogger which 
				simply records any and all keypresses by the user.
			\item Trojan horse: Software that is meant to look exactly like another
				piece of software, once executed, releases other forms of malware onto
				the device.
			\item Adware: Malware that automatically pops up advertisements, which is
				an irritating experience.
			\item Ransomware: Encrypts user's data and asks for money in exchange for
				decryption of that data.
		\end{itemize}
	\item Pharming: Used to obtain user's personal data. Once malware is installed,
		the user is sent to websites that look very much like the websites the user is
		going to but is actually set up by the perpetrator. The user then unknowingly
		enters their credentials, which are now known by the perpetrator.
	\item Phishing: The objective is same as pharming, only the perpetrator tricks the
		user to click a link, and the rest is same as pharming.
	\item Social engineering: Scammers, who trick users by lying, acting and decieving
		them.
\end{itemize}
\subsubsection{Solutions}
To save oneself from these threats we can use the following measures:
\begin{itemize}
	\item Access levels: Only allow certain data to be accessible by users with 
		administrator or higher levels.
	\item Anti-malware: Software that detects and reports malicious activity by 
		software.
	\item Authentication: Setting very strong passwords, that are very difficult to 
		guess, including unique characters such as numbers, varying cases in letters,
		etc. Biometrics can be used to make sure data can only be accessed by only the
		user with specific biological signatures such as fingerprints and retina scans.
		Lastly two-step verification can be used to make sure that even if perpetrators
		gain access to username and password, they require something that only you have
		to log into your account.
	\item Software updates: Outdated software may have vulnerabilities which hackers
		can exploit, to avoid this, one can automate the updating of software.
	\item Spelling and tone: To avoid being phished or pharmed, check email tone and
		spelling properly.
	\item Check attached URL: A link may say "to know more" but the URL behind attached
		may be something else. Check them to be safe.
	\item Firewalls: Software that examines incoming and outgoing data across a network.
		This can detect malicious transfers of data and report and block those 
		transmissions.
	\item Proxy servers: Can be used by web servers to examine requests sent before
		they are sent to the actual server. As a result web servers are protected from
		DDoS attacks.
	\item Secure Sockets Layer (SSL): HTTPS uses the SSL protocol.
\end{itemize}

\section{Automated and emerging technologies}
\subsection{Automated systems}
\subsubsection{Sensors, microprocessors and actuators}
Inputs can be taken by sensors, processed (compared with stored values) by 
microprocessors and actions can be taken by actuators.
	
\subsubsection{Automation in context}
To consider the advantages and disadvantages of automated systems in given contexts,
one can consider the following factors:
\begin{itemize}
	\item Initial cost: Can be high to develop and install the system.
	\item Running cost: Will be low as employees need not be paid.
	\item Safety: Will be high as people need not work in dangerous places, people
		aren't working and so accidents are unlikely.
	\item Replacing people's jobs: Jobs automated by the system will be lost but more
		will be made in maintaining and operating the system.
	\item Continuous work: If the context requires 24/7 operation, it is an advantage.
	\item Precision: Of course automated devices are more precise.
\end{itemize}
\subsection{Robotics}
\subsubsection{Characteristics of a robot}
A robot consists of a physical, mechanical framework. Sensors, microprocessors and
actuators and lastly they are programmable meaning they can be told to follow 
instructions.
\subsubsection{Robotics in context}
Identical to Automation in context.

\subsection{Artificial intelligence (AI)}
\subsubsection{What is it?}
It is a branch of computer science dealing with the simulation of intelligent behaviours
by computers.
\subsubsection{Characteristics for AI}
Data must be collected and given to the AI so that it can reason, it must programmed
with rules so that it can reason, the ability to reason and lastly the ability to learn
from inputs given by users and hence adapt.

\subsubsection{Machine learning and expert systems}
Machine learning is when a system learns about information either by itself
(unsupervised) or is told about information (supervised). A machine is given a picture
of a horse and is told it is a picture of a horse, this is supervised, unsupervised
may involve graph plotting.

Expert systems consist of a knowledge base which is a list of facts; a rule base which
are facts that link the facts, an inference engine which decides what to ask next and
when to ask it and lastly the user interface so that the user can communicate with the
system.
\end{document}
